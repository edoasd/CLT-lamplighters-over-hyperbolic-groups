\section{Preliminaries}
\subsection{Notation for graphs and paths}
\begin{itemize}
	\item We will work with undirected, unlabeled graphs $(V,E)$.
	\item A path $\gamma=(\gamma_0,\gamma_1,\ldots, \gamma_n)$ is an ordered sequence of vertices in the graph.
\end{itemize}

\subsection{Hyperbolic groups}
Say basic things about hyperbolicity; explain pivots

\subsection{Wreath products}
We consider the wreath products $A\wr H$, where $A$ is a finite non-trivial group and $H$ is a finitely generated group. Let $S_H$ be a finite and symmetric generating set of $H$. 

\subsubsection{The switch-walk-switch word metric}\label{subsubsection: the sws word metric}
We consider the \emph{switch-walk-switch} $S_{\mathrm{sws}}$ generating set of $A\wr H$, given by
\[
S_{\mathrm{sws}}\coloneqq \Big\{ (\delta_a,0)(\mathbf{0},s)(\delta_{a^{\prime}},0)\Big| a,a^{\prime}\in A\text{ and }s\in S_H \Big\}.
\]

\begin{thm}[{\cite[Theorem 1.2]{Parry1992}}]
For any $g=(f,x)\in A\wr H$, the word length of $g$ with respect to the standard generating set is
	\[
	|g|_{\mathrm{sws}}=\mathrm{TSP}(e_H,x,\supp{f}).
	\]
\end{thm}


\subsection{Random walks on groups}
 Let $G$ be a countable group and consider a probability measure $\mu$ on $G$. Consider the product space $\Omega\coloneqq G^{\Z_{+}}$ endowed with the product $\sigma$-field. For each $n\ge 1$ we denote by
	\begin{equation*}
		\begin{aligned}
			X_n:\Omega&\to G\\
			w\coloneqq(w_1,w_2,\cdots)&\mapsto X_n(w)\coloneqq w_n
	\end{aligned}
	\end{equation*}
	the $n$-th coordinate map. We endow $\Omega$ with the product probability measure $\mu^{\Z_{+}}.$
		
We denote by
	\begin{equation*}
	\begin{aligned}
		\theta:\Omega&\to \Omega\\
		w\coloneqq(w_1,w_2,\cdots)&\mapsto \theta(w)\coloneqq (w_2,w_3,\ldots)
	\end{aligned}
\end{equation*}
the shift map in the space of increments.


Now we define the $\mu$-random walk $\{Z_n\}_{n\ge 0}$ on $G$ as follows. We define $Z_0(w)=e_G$ for each $w\in \Omega$, and for each $n\ge 1$ we define
\[
Z_n(w)\coloneqq Z_{n-1}(w)\cdot X_n(w).
\]
We remark that $Z_n(w) (Z_m\circ \theta^n)(w)=Z_{n+m}(w)$, for each $w\in \Omega$ and $n,m\ge 1$.


\subsection{Defective adapted cocycles and the central limit theorem}

A sequence $\mathcal{Q}=\{Q_n\}_{n\ge 1}$ of maps $Q_n:\Omega\to \R$ such that $Q_n$ is measurable with respect to $\sigma(X_1,\ldots, X_n)$, for each $n\ge 1$, is called a \emph{defective adapted cocycle}. We will use the convention $Q_0\equiv 0.$ The \emph{defect of $\mathcal{Q}$} is the collection of maps $\Psi=\{\Psi_{n,m}\}_{n,m\ge 0}$ defined by
\[
\Psi_{n,m}(w)=Q_{n+m}(w)-Q_n(w)-(Q_m\circ \theta^n)(w), \text{ for each }w\in \Omega \text{ and }n,m\ge 0.
\] 
The following result states that the central limit theorem holds for defective adapted cocycles that satisfy a second-moment deviation inequality.


\begin{thm}[{\cite[Theorem 4.2]{MathieuSisto2020}}] \label{thm: MS general CLT with constant deviation ineq}
	Let $G$ be a countable group endowed with a probability measure $\mu$. Consider $\mathcal{Q}$ a defective adapted cocycle on $\Omega=G^{\Z_{+}}$, and denote by $\{\Psi_{n,m}\}_{n,m\ge 0}$ its defect. Suppose that
	\begin{enumerate}
		\item $\E[|Q_1|^{2}]<\infty$, and
		\item 	$\sup_{m,n\ge 0} \left\{ \E\left[|\Psi_{n,m}|^2\right]\right\} <\infty$.
	\end{enumerate}
	
Then, there exist constants $\ell,\sigma\in \R$ such that the random variables $\frac{1}{\sqrt{n}}\left(Q_n-\ell n\right)$ converge in law to a Gaussian random variable with zero mean and variance $\sigma^2$.
\end{thm}

Furthermore, it is proved in \cite[Theorem 3.3]{MathieuSisto2020} that the constant $\ell$ that appears in the statement of Theorem \ref{thm: MS general CLT with constant deviation ineq} satisfies that $\frac{1}{n}Q_n$ converges to $\ell$ in $L_1$ as $n\to \infty$.  


We will use this result for the defective adapted cocycle obtained from the word length of the $\mu$-random walk on $G$ at time $n$. That is, we will consider some word metric $d$ on $G$, and define $Q_n\coloneqq d(e_G,Z_n)$, for each $n\ge 1$. Since we will be working with finitely supported probability measures, it holds immediately that $\E[|Q_1|^2]<\infty$. 


The objective of the following sections of this paper is to prove that there exists a constant $C>0$ such that
\[
\E[d(e_G,Z_{n+m})-d(e_G,Z_n)-d(e_G,Z_m\circ \theta^n)]=\E[|\Psi_{n,m}|^2]\le C, \text{ for each }n,m\ge 0,
\]

where $G=A\wr H$, $d$ will be the switch-walk-switch word metric and $\mu$ is a finitely supported probability measure, as in the hypotheses of Theorem \ref{thm: main theorem CLT for word length for lamplighter over a hyperbolic base group}.
