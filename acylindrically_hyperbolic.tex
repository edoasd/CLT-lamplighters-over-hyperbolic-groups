
\section{The CLT for lamplighter random walks on acylindrically hyperbolic groups}

\begin{thm} \label{thm:lamplighterCLT}
    Let $G = \Z/2\Z \wr H$ be 
    a wreath product over a Gromov-hyperbolic group $H$. Let $\tilde{Z}_n$
    be a random walk on $G$ with 
    step distribution which is finitely supported and \textcolor{red}{[what is the term for these types of generators? where everything is step or step+light]}. \textcolor{blue}{switch-walk-switch?}
    Then $d(o,\tilde{Z}_n)$ satisfies a CLT.
\end{thm}

\textcolor{red}{(Maybe adopt the convention that $\tilde{Z}_n$ is the random walk on the wreath product $G$, while $Z_n$ is its projection to the hyperbolic base group $H$. I think that the notation we've been using is that $R_n \subset H$ is the configuration of lights at time $n$, i.e.
$\tilde{Z}_n = (R_n, Z_n)$. Since this looks like the range of the random walk, maybe we want to change notation, but this is the notation I'm going to use as I write the lemmas we need for now.)}

For $R \subset H$, $x,y \in H$,
denote by $TSP(x,R,y)$ a solution to the traveling salesman problem, that is, the edge path in the Cayley graph of $H$ which starts at $x$, ends at $y$, visits each vertex in $R$, and has minimal length subject to this constraints.
Note that for our choice of generators for $G$, we have that $d((\emptyset,1),(R,x)) = |TSP(1,R,x)|$.

%\begin{thm}[{\cite{MathieuSisto2020}}] \label{thm:generalCLT}
%    Suppose that $Q_n$ is a defective adapted cocycle with defect
%    \[
%       \Phi_{m,n} := Q_n - (Q_m + Z_m Q_{m-n})
%    \]
%    and suppose that for some fixed polynomial $p$ and $N_0 \in \mathbb{N}$ we have that
%    \[
%       \E[|\Phi_{m,n}|^2] \le p(\log n)
%    \]
%    whenever $m, n-m \ge N_0$.
%    Then a CLT holds for $Q_n$.
%\end{thm}

\begin{lem}[Deterministic TSP comparison] \label{lem:deterministic}
   \textcolor{red}{When we write this out properly, we'll figure out what all the proper polynomials are, and the proper definition of the ``middle region'', etc.}
   There are fixed polynomials $p_1,p_2,p_3,p_4$ such that the following hold.
   Let $C_1, C_2,$ and $C_3$ be
   constants. 
   Let $(R_m,x_m), (R_n,x_n) \in G$
   and define $R_{m,n} \subset H$
   by 
   $(R_{m,n}, x_m^{-1} x_n) :=
   (R_m, x_m)^{-1} (R_n, x_n)$.
   Suppose that
   \begin{enumerate}
       \item $R_m \cup x_m R_{m,n} \subset \mathcal{N}(\xi, C_1 p_1(\log n))$,
       that is,
       every vertex of $R_m \cup x_m R_{m,n}$ is within distance 
       $C_1 p_1(\log n)$ of $\xi$, 
       where $\xi$ is a geodesic
       from $1$ to $x_n$.
       \item Setting $r := d(o,x_m)$,
       we have 
       $R_m \subset B(o,r+C_2 p_2(\log n))$ and
       $x_m R_{m,n} \subset
       B(o, r - C_2 p_2(\log n))^c$.
       \textcolor{red}{(Not exactly
       clear yet what will be the best way to define the ``middle/left/right'' of the path yet, but I'm using this definition at least for now since it probably will come most easily out of the Hoeffding speed concentration estimates.}
       \item Set
       \[
          M := B(o,r+C_2 p_2(\log n)) \cap B(o, r - C_2 p_2(\log n))^c.
       \]
       Then we have that
       \[
          |M \cap (R_m \cup x_m R_{n,m})| \le C_3 p_3(\log n).
       \]
   \end{enumerate}
   Then we have that
   \begin{align*}
      &|TSP(1,R_n,x_n)| \ge \\
      &|TSP(1,R_m,x_m)| +
      |TSP(1,R_{m,n}, x_m^{-1} x_n)|
      - (C_1 + C_2 + C_3)p_4(\log n).
   \end{align*}
   \textcolor{red}{Depending on the analysis, could maybe get some 
   different expression in the 
   $C_i$. If we have something
   of this form (or equivalently,
   a bound which is poly(log) times
   $\max(C_1, C_2, C_3)$),
   then to get the final step
   it will suffice to show the each
   $C_i$ is square-integrable (as 
   a random variable), but
   if other powers of the $C_i$
   become involved, might
   have to show stronger
   integrability.}
\end{lem}

\begin{proof}
The outline of the proof is as follows.
To simplify the notation, let $D=p_1(\log n)$. Notice that by quasi-convexity  there is an absolute constant $\delta$ such that for any two points $x,y \in N(\xi, D)$ any geodesic between $x$ and $y$ will be contained in $N(\xi,D+\delta)$.  



\begin{enumerate}
    \item  We will split $N(\xi, D)$ into three regions - the initial part $I$ close to the origin, the central region $C$ and the terminal part close to $x_n$.  More precisely,  we let $I:=N(\xi, D)\cap B(o, r - C_2 p_2(\log n)$, $C:= N(\xi,D) \cap B(o,r+C_2 p_2(\log n)) \cap B(o, r - C_2 p_2(\log n))^c$ and $T=N(\xi,D) \cap B(o, r + C_2 p_2(\log n))^c$. Polynomial $p_2$ will be determined later.
    
    \item  
    Pick any solution $S$ to  $TSP(1,R_n,x_n)$.  We are going to show that it is possible to add a collection of segments of total length bounded by a fixed polynomial in $\log_n$ to $\overline{I \cap S}$  to get a path starting at $1$, visiting every point in $R_m$ and ending at $x_m$.  Here  $\overline{I \cap S}$ stands for the union of $I \cap S$  together with all geodesic segments of $S$ that join two points in $R_n  \cap I $  but possibly go outside of $I$. Then by the definition of TSP this path would have the length bounded below by $|TSP(1,R_m,x_m)|$.  
    
    Similar argument will be applied to $\overline{S \cap T}$ and $|TSP(1,R_{m,n}, x_m^{-1} x_n)|$. Since $\overline{I \cap S}$ and $\overline{T \cap S}$ have no geodesic segments of $S$ in common,  we have  
    \begin{align*}
    &|TSP(1,R_n,x_n)|= |S| \ge |\overline{I \cap S}|+|\overline{T \cap S}| \ge \\ 
     &|TSP(1,R_m,x_m)| +
      |TSP(1,R_{m,n}, x_m^{-1} x_n)| -P(\log n)
    \end{align*}
\end{enumerate}
\end{proof}


Since $\Psi_{n,m}$ is bounded above by $3n$, we should be able to reduce the problem to deterministic case using  known deviation estimates and a simple union bound, so most likely we will not need the strongest versions of the next lemmas.

\begin{lem} \label{lem:boundC1}
    Define the random variable $\mathcal{C}_{1,m,n}$ to be the
    minimal value of $C_1$
    such that (1) in Lemma \ref{lem:deterministic} holds 
    for $\tilde{Z}_m = (R_m,x_m)$
    and $\tilde{Z}_n = (R_n,x_n)$,
    i.e.
    \[
       \mathcal{C}_{1,m,n} := 
       \inf \{ C_1 \ge 0 : 
       d(r,\xi) \le C_1 p_1(\log n) 
       \mbox{ for all } 
       r \in R_m \cup Z_m R_{n,m} \},
    \]
    where $\xi$ is a geodesic
    from $1$ to $x_n = Z_n$.
    Then
    \[
       \limsup_{m,n-m \to \infty} \E[ \mathcal{C}_{1,m,n}^2 ] < \infty.
    \]
    \textcolor{red}{The proof of this should be via a tail bound
    which should be contained in ``Tracking Rates of Random Walks'' by Sisto.}
\end{lem}
 Since $\Psi_{n,m}$ is bounded above by $3n$, we should be able to reduce the problem to deterministic case using  known deviation estimates and a simple union bound, so most likely we will not need the strongest versions of the next lemmas.
\begin{lem} \label{lem:boundC2}
    Define the random variable $\mathcal{C}_{2,m,n}$ to be the
    minimal value of $C_2$
    such that (2) in Lemma \ref{lem:deterministic} holds 
    for $\tilde{Z}_m = (R_m,x_m)$
    and $\tilde{Z}_n = (R_n,x_n)$.
    Then
    \[
       \limsup_{m,n-m \to \infty} \E[ \mathcal{C}_{2,m,n}^2 ] < \infty.
    \]
    \textcolor{red}{It seems likely enough that in the end the bound on $C_2$ will come simultaneously with the bound on either $C_1$ or $C_3$, since it
    should come from either tracking or speed. In fact, since $C_2$ is a parameter that we use to define the ``middle'' segment $M$, we will want to make it \emph{bigger} than the minimum possible. Maybe in the final argument we will just take 
    $C_2 = 6 C_1$ or $6 C_3$ or something like that and this lemma won't exist.}
\end{lem}

\begin{lem} \label{lem:boundC3}
    Define the random variable $\mathcal{C}_{3,m,n}$ to be the
    minimal value of $C_3$
    such that (3) in Lemma \ref{lem:deterministic} holds 
    for $\tilde{Z}_m = (R_m,x_m)$
    and $\tilde{Z}_n = (R_n,x_n)$.
    That is,
    \[
       \mathcal{C}_{3,m,n} :=
       \frac{|M \cap (R_m \cup x_m R_{m,n})|}{p_3(\log n)}.
    \]
    Then
    \[
       \limsup_{m,n-m \to \infty} \E[ \mathcal{C}_{3,m,n}^2 ] < \infty.
    \]
    \textcolor{red}{I expect this to follow from the concentration bounds for speed in ``Random walks in hyperbolic spaces...'' by Aoun and Sert. Probably not quite as immediate as bound on tail of $\mathcal{C}_1$,
    but I think should go through.}
\end{lem}

\begin{proof}[Proof of Theorem \ref{thm:lamplighterCLT} given Lemmas]
   \textcolor{red}{Again, this
   is assuming that the conclusion
   of the deterministic lemma
   has the form I wrote;
   write out that lemma
   carefully and adjust the 
   proof here/lemmas above as 
   needed.}
   Define a defective adapted
   cocycle by
   \[
      Q_n := |TSP(1,R_n,Z_n)|
   \]
   where $(R_n,Z_n) = \tilde{Z}_n$.
   By construction of the $\mathcal{C}_i$ and by
   Lemma \ref{lem:deterministic},
   we have that the defect satisfies
   \[
      |\Psi_{m,n}| \le
      (\mathcal{C}_{1,m,n} +
      \mathcal{C}_{2,m,n} +
      \mathcal{C}_{3,m,n})p_4(\log n).
   \]
   By Lemmas \ref{lem:boundC1},
   \ref{lem:boundC2}, and \ref{lem:boundC3}, there 
   is some constant $C$ (independent
   of $m$ and $n$) and some $N_0 \in \N$ such that
   whenever $m, m-n \ge N_0$
   we have that
   \[
      \E[ \mathcal{C}_{i,m,n}^2] \le C
   \]
   for $i=1,2,3$. Cauchy-Schwarz
   then tells us that for some $C'$,
   whenever $m, n-m \ge N_0$ we 
   have
   \[
      \E[|\Psi_{m,n}|^2] \le C' p_4(\log n)^2.
   \]
   Then applying Theorem \ref{thm: polylogarithmic MS CLT} gives our result.
\end{proof}