\documentclass[reqno,oneside,11pt]{amsart}

\usepackage[T1]{fontenc}
\usepackage[utf8]{inputenc}
\usepackage{lmodern}
\usepackage[french,english]{babel}
\usepackage{geometry} 
%\geometry{a4paper}

\geometry{%
	a4paper,                % format de papier
	% Définition des marges :
	left= 3cm,            % marge intérieure à la page
	right = 3cm,          % marge extérieure
	top = 3cm,
	bottom = 3cm,
	% En-tête et pied de page :
	headheight=6mm,         % espace réservé à l'en-tête dans la marge top
	headsep=9mm,            % espace entre le corps et l'en-tête
	footskip=9mm            % espace entre le corps et le pied de page
}
\usepackage{amsmath,amssymb,amsfonts,amsthm,mathtools}
\usepackage{mathrsfs}
\usepackage{graphicx}
\usepackage{caption}
\usepackage{subcaption} 
\usepackage{booktabs}
\usepackage{array}
\usepackage{paralist}
\usepackage{fancyhdr}
\usepackage{emptypage}
\pagestyle{fancy}
\renewcommand{\sectionmark}[1]{\markright{\thesection.\ #1}}
\fancyhf{}
\fancyhead[LE]{\leftmark}
\fancyhead[RO]{\rightmark}
\fancyfoot[C]{\thepage}

\fancypagestyle{plain}{
	\fancyhf{}
	\fancyfoot[RO,RE]{\thepage}
	\renewcommand{\headrulewidth}{0pt}
	\renewcommand{\footrulewidth}{0pt}}
\newcommand*{\img}[1]{%
	\raisebox{-.0\baselineskip}{%
		\includegraphics[
		height=\baselineskip,
		width=\baselineskip,
		keepaspectratio,
		]{#1}%
	}%
}
\newcommand\blfootnote[1]{%
	\begingroup
	\renewcommand\thefootnote{}\footnote{#1}%
	\addtocounter{footnote}{-1}%
	\endgroup
}

\usepackage{comment} 
\usepackage{diagbox}
\usepackage{xcolor}
\usepackage{epigraph}

\usepackage[noadjust]{cite}
\usepackage[
pagebackref=true,
colorlinks=true,
urlcolor=purple,
linkcolor=purple!87!black,
citecolor=green!60!black,
pdfborder={0 0 0}
]{hyperref}
\renewcommand*{\backref}[1]{}
\renewcommand*{\backrefalt}[4]{[{\tiny%
		\ifcase #1 Not cited.%
		\or Cited on page~#2.%
		\else Cited on pages #2.%
		\fi%
	}]}
\usepackage{bookmark}
\usepackage{dsfont}
%% Colors
\newcommand{\red}[1]{\textcolor{red}{#1}}
\newcommand{\blue}[1]{\textcolor{blue!60!black}{#1}}
%% Common notation
\newcommand{\id}{\mathsf{id}}
\newcommand{\supp}[1]{\mathrm{supp}(#1)}
\newcommand{\cay}[2]{\mathrm{Cay}(#1,#2)}
\newcommand{\hceil}[1]{\Big\lceil #1 \Big\rceil}
\newcommand{\hfloor}[1]{\Big\lfloor #1 \Big\rfloor}
\newcommand{\BS}{\mathrm{BS}}
%% Numbers: Z, R, C, Q
\newcommand{\Z}{\mathbb{Z}}
\newcommand{\R}{\mathbb{R}}
\newcommand{\C}{\mathbb{C}}
\newcommand{\Q}{\mathbb{Q}}
%% Probabilty: Proba, expectation
\renewcommand{\P}{\mathbb{P}}
\newcommand{\E}{\mathbb{E}}
%% TIKZ
\usepackage{tikz}
\usetikzlibrary{patterns,positioning,arrows,decorations.markings,calc,decorations.pathmorphing,decorations.pathreplacing}
\usetikzlibrary{fadings}
\usepackage{forest}
\usepackage{tikz-qtree}
\RequirePackage{pgffor} 
\makeatletter
\newcommand{\neutralize}[1]{\expandafter\let\csname c@#1\endcsname\count@}
\makeatother

\newenvironment{questionbis}[1]
{\addtocounter{thm}{-1}
	\renewcommand{\thethm}{\ref*{#1}$'$}%
	\neutralize{question}\phantomsection
	\begin{question}}
	{\end{question}}

\newcommand{\thistheoremname}{}

\newtheorem*{genericthm*}{\thistheoremname}
\newenvironment{namedthm*}[1]
{\renewcommand{\thistheoremname}{#1}%
	\begin{genericthm*}}
	{\end{genericthm*}}
\theoremstyle{plain}
\newtheorem{thm}{Theorem}[section] % reset theorem numbering for each chapter
\newtheorem{prop}[thm]{Proposition}
\newtheorem{lem}[thm]{Lemma}
\newtheorem{cor}[thm]{Corollary}
\theoremstyle{definition}
\newtheorem{rem}[thm]{Remark}
\newtheorem{defn}[thm]{Definition} % definition numbers are dependent on theorem numbers
\newtheorem{exmp}[thm]{Example} % same for example numbers
\newtheorem{question}[thm]{Question}
\newtheorem{conjecture}[thm]{Conjecture}
\newtheorem{exercise}[thm]{Exercise}
\newtheorem{condition}{Condition}


\title{CLT for range of random walks on hyperbolic groups}
\author{Maksym Chaudkhari, Kunal Chawla, Christian Gorski, Eduardo Silva }
\date{\today}

%%%%%%%%%%%%%%%%%%%%%%%%%%%%%%%%%%%%%%%%%%%%%%%%%%%%%%%%%%%%%%%%%%%%%%%%%%%%%%%%%%%%%%%%%%%%%%%%
%
% Hi everybody! 
%I'm trying to uniformize notation and start to make the paper look closer to a "complete" draft. I will start by trying to write down well the preliminaries and notation associated to Matthieu Sisto, and to order the argument in the case of hyperbolic groups that uses pivots (since it is shorter and I think it is a nice first "checkpoint" for the paper). Once we have that I think it will be easier to complete the paper to include the case of acylindrically hyperbolic groups and the argument with the polylog deviation inequalities. Let me know if you have any comments about something that I wrote!

%Eduardo.
%
%
%%%%%%%%%%%%%%%%%%%%%%%%%%%%%%%%%%%%%%%%%%%%%%%%%%%%%%%%%%%%%%%%%%%%%%%%%%%%%%%%%%%%%%%%%%%%%%%%

\begin{document}
	\begin{abstract} 
We prove a central limit theorem for random walks with a finitely supported step distribution on wreath products of the form $A\wr H=\bigoplus_H A\wr H $, where $A$ is a non-trivial finite group and $H$ is a non-elementary hyperbolic group.
	\end{abstract}
\maketitle

\section{Introduction}



\begin{enumerate}
\item	Talk about random walks on groups; trying to prove limit laws in general
\item Make a list of things known about lamplighters and why they are relevant
\item Say that in this paper we concentrate on the CLT
\end{enumerate}
\blue{Here are some things that we should cite... I am missing many more but this is a good start.}
\cite{Salaun2001}.

\begin{enumerate}
	\item The CLT for non-abelian free groups is due to \cite{SawyerSteger1987} and \cite{Ledrappier2001}. Then for non-elementary hyperbolic groups with a finite exponential moment is due to \cite{Bjorklund2010}. This was generalized for any finite second moment measure in \cite{BenoistQuint2016hyperbolic}. The last two results hold more generally for group acting on a Gromov hyperbolic space by isometries. \cite{BenoistQuint2016} show a CLT for random walks on $\mathrm{GL}_d(\R)$ with a finite second moment. See also \cite{Gouezel2017}.
	\item \cite{ErschlerZheng2022} prove a law of large numbers for random walks on $\Z/2\Z\wr \Z^2$ with a finite $(2+\varepsilon)$-moment, for some $\varepsilon>0$. They also discuss limit laws in other wreath products.
	\item \cite{BahmanianForghaniGekhtmanMallahiKarai2024} prove a central limit theorem for random walks on the group of affine transformations of a horospherical product of Gromov hyperbolic spaces.
	\item \cite{Choi2023} proves a central limit theorem for groups acting with contracting elements.
	\item \cite{GekhtmanTaylorTiozzo2022} prove a CLT with respect to the counting measure on the Cayley graph of a group acting on a hyperbolic space.
	\item \cite{Horbez2018} proves a CLT for random walks on mapping class groups and $\mathrm{Out}(F_n).$
	\item \cite{LeBars2022} proves a CLT for groups acting on a $\mathrm{CAT}(0)$ space.
	\item  \cite{Gilch2008} proves that the drift of $\Z/2\Z\wr G$ is strictly larger than that of its projection to $G$.
	\item \cite{MrazovicSandriSebek2023} prove a LLN and CLT for the capacity of the range of a random walk on a group.
	\item \cite{Salaun2001} proves a LLN and CLT for a simple random walk on a free group, conditioned on the boundary point.
\end{enumerate}


\subsection{Main results}

Consider the switch-walk-switch word length $|\cdot |$ on $A\wr H$.

\begin{thm}
	Let $A$ be a non-trivial group and $H$ a non-elementary hyperbolic group. Consider a probability measure $\mu$ on $A\wr H$ such that $\mu_H$ is non-elementary and has a finite second moment, and such that $\mu$ has bounded lamp range. Denote by $\{w_n\}_{n\ge 0}$ the $\mu$-random walk on $A\wr H$, and let $C=\lim_{n\to \infty}\frac{\E(|w_n|)}{n}$ be the drift of the $\mu$-random walk on $A\wr B$. Then the sequence of normalized random variables $\frac{|w_n|-Cn}{\sqrt{n}}$, $n\ge 1$, converges in law to a non-degenerate gaussian law.
\end{thm}

\red{What are the moment hypotheses for acylindrically hyperbolic base groups? Finite support? exponential tails?}

We prove a central limit theorem for the range of random walks with a finite second moment on hyperbolic groups.
\begin{thm}\label{thm: CLT range on hyperbolic groups}
	Let $H$ be a non-elementary hyperbolic group and let $\mu$ be a non-elementary probability measure on $H$ with a finite second moment. Let $C$ be the probability that the $\mu$-random walk on $H$ starting at $e_H$ never returns to $e_H$. Then the sequence of normalized random variables $\frac{|R _{n} | - Cn}{\sqrt{n}}$, $n\ge 1$, converges in law to a non-degenerate gaussian law.
\end{thm}

\section{Preliminaries}

\subsection{Hyperbolic groups}
Say basic things about hyperbolicity; explain pivots

\subsection{Lamplighter groups}
We consider the wreath products $A\wr H$, where $A$ is a finite non-trivial group and $H$ is a finitely generated group. Let $S_H$ be a finite and symmetric generating set of $H$. Then we consider the \emph{switch-walk-switch} $S_{\mathrm{sws}}$ generating set of $A\wr H$, given by
\[
S_{\mathrm{sws}}\coloneqq \Big\{ (\delta_a,0)(\mathbf{0},s)(\delta_{a^{\prime}},0)\Big| a,a^{\prime}\in A\text{ and }s\in S_H \Big\}.
\]

\begin{thm}[{\cite[Theorem 1.2]{Parry1992}}]
For any $g=(f,x)\in A\wr H$, the word length of $g$ with respect to the standard generating set is
	\[
	|g|=\mathrm{TSP}(e_H,x,\supp{f}).
	\]
\end{thm}


\subsection{Random walks on groups}
\begin{enumerate}
	\item Recall basic concepts of random walks on groups.
\end{enumerate}
\subsection{Defective adapted cocycles and the central limit theorem}

\begin{enumerate}
	\item Introduce all necessary results and definitions of Mathieu-Sisto.
	\item Introduce pivots and the results of Gouëzel that we will use.
\end{enumerate}


We will use the following general criterion of Mathieu-Sisto.

\begin{defin}[Defective adapted cocycle]

\end{defin}

\begin{thm}[\cite{MathieuSisto2020}] \label{thm: MS general CLT with constant deviation ineq}
	Suppose that $Q_n$ is a defective adapted cocycle with defect
	\[
	\Phi_{m,n} := Q_n - (Q_m + w_m Q_{m-n}).
	\]
	Suppose that there exists $C>0$ such that  \[
	\sup_{m,n\ge 1}\E[|\Phi_{m,n}|^2] \le C.
	\]
	 Then the CLT holds for $Q_n$.
\end{thm}

\section{Mathieu-Sisto's deviation inequalities and consequences}
\subsection{Defective adapted cocycles and the central limit theorem}

A sequence $\mathcal{Q}=\{Q_n\}_{n\ge 1}$ of maps $Q_n:\Omega\to \R$ such that $Q_n$ is measurable with respect to $\sigma(X_1,\ldots, X_n)$, for each $n\ge 1$, is called a \emph{defective adapted cocycle}. We will use the convention $Q_0\equiv 0.$ The \emph{defect of $\mathcal{Q}$} is the collection of maps $\Psi=\{\Psi_{n,m}\}_{n,m\ge 0}$ defined by
\[
\Psi_{n,m}(w)=Q_{n+m}(w)-Q_n(w)-(Q_m\circ \theta^n)(w), \text{ for each }w\in \Omega \text{ and }n,m\ge 0.
\] 
The following result states that the central limit theorem holds for defective adapted cocycles that satisfy a second-moment deviation inequality.


\begin{thm}[{\cite[Theorem 4.2]{MathieuSisto2020}}] \label{thm: MS general CLT with constant deviation ineq}
	Let $G$ be a countable group endowed with a probability measure $\mu$. Consider $\mathcal{Q}$ a defective adapted cocycle on $\Omega=G^{\Z_{+}}$, and denote by $\{\Psi_{n,m}\}_{n,m\ge 0}$ its defect. Suppose that
	\begin{enumerate}
		\item $\E[|Q_1|^{2}]<\infty$, and
		\item 	$\sup_{m,n\ge 0} \left\{ \E\left[|\Psi_{n,m}|^2\right]\right\} <\infty$.
	\end{enumerate}
	
	Then, there exist constants $\ell,\sigma\in \R$ such that the random variables $\frac{1}{\sqrt{n}}\left(Q_n-\ell n\right)$ converge in law to a Gaussian random variable with zero mean and variance $\sigma^2$.
\end{thm}

Furthermore, it is proved in \cite[Theorem 3.3]{MathieuSisto2020} that the constant $\ell$ that appears in the statement of Theorem \ref{thm: MS general CLT with constant deviation ineq} satisfies that $\frac{1}{n}Q_n$ converges to $\ell$ in $L_1$ as $n\to \infty$.  


We will use this result for the defective adapted cocycle obtained from the word length of the $\mu$-random walk on $G$ at time $n$. That is, we will consider some word metric $d$ on $G$, and define $Q_n\coloneqq d(e_G,Z_n)$, for each $n\ge 1$. Since we will be working with finitely supported probability measures, it holds immediately that $\E[|Q_1|^2]<\infty$. 


The objective of the following sections of this paper is to prove that there exists a constant $C>0$ such that
\[
\E[d(e_G,Z_{n+m})-d(e_G,Z_n)-d(e_G,Z_m\circ \theta^n)]=\E[|\Psi_{n,m}|^2]\le C, \text{ for each }n,m\ge 0,
\]

where $G=A\wr H$, $d$ will be the switch-walk-switch word metric and $\mu$ is a finitely supported probability measure, as in the hypotheses of Theorem \ref{thm: main theorem CLT for word length for lamplighter over a hyperbolic base group}.

\subsection{Continuity results}
The following results are obtained in Sections 5 and 6 of Mathieu-Sisto paper.
\begin{enumerate}
	\item Continuity and differentiability of drift .
	\item Continuity and differentiability of asymptotic entropy.
\end{enumerate}
We will apply them to lamplighters over hyperbolic groups.
\subsection{Polylog deviation inequalities}

In this section we explain a generalization of \cite[Theorems 4.1 \& 4.2]{MathieuSisto2020} (see Theorem \ref{thm: MS general CLT with constant deviation ineq})
\red{remember to do the modification of n to m+n.}
\begin{thm} \label{thm: polylogarithmic MS CLT}
	Suppose that $Q_n$ is a defective adapted cocycle with defect
\[
\Psi_{n,m}(w)=Q_{n+m}(w)-Q_n(w)-(Q_m\circ \theta^n)(w), \text{ for each }w\in \Omega \text{ and }n,m\ge 0.
\] 
	Suppose that for some fixed polynomial $p$ and $N_0 \in \mathbb{N}$ we have that
	\[
	\E[|\Psi_{n,m}|^2] \le p(\log (n+m))
	\]
	whenever $n,m \ge N_0$.
	Then a CLT holds for $Q_n$.
\end{thm}

Here are the places in the proof of Mathieu-Sisto's CLT where the deviation inequality is used:
\begin{enumerate}
%	\item In Theorem 4.4 one obtains $V(Q_n)\le n\Big(p(4 \E(Q_1^2)+ 16\log(n))\Big)$
%	\item Then in Lemma 4.5 it is cited a result of Hammersley \cite[Theorem 2]{Hammerslet1962}. This should be replaced by an inequality of the form $a_{n+m}\le a_n+a_m+bp(log(n+m))$.
	\item The deviation inequality is used in multiple occasions during the proof of Lemma 4.6. One should do the appropriate modifications.
\end{enumerate}
The proof follows very similarly to that of \cite[Theorems 4.1\& 4.2]{MathieuSisto2020}. We now explain the parts of the proof where minor modifications are needed.

	The first step in the proof is \cite[Theorem 4.4]{MathieuSisto2020} and its application to obtain \cite[Theorem 4.2]{MathieuSisto2020}. With an analogous proof, one obtains 
	\[
	V(Q_n)\le n\left(p\E(Q_1^2)+16\log(n)\right),
	\]
	where $p$ is the polynomial from the second moment deviation inequality.
	
	The second part of the proof is in \cite[Lemma 4.5]{MathieuSisto2020}. Here, we can use \cite[Theorem 2]{Hammerslet1962} to obtain the following statement.
	\begin{lem}\label{lem: hammersley}
		Let $(a_n)_{n\ge 0}$ be a sequence of real numbers and let $p$ be a polynomial. Suppose that there is $b\ge 0$ such that
		\[
		a_{n+m}\le a_n+a_m+bp(\log(n+m)),\text{ for each }n,m\ge 1.
		\]
		Then $\lim_{n\to \infty}\frac{a_n}{n}$ exists.
	\end{lem}
	 The most important part of the proof is the following lemma. 
	 \begin{lem}
	 	Let $(Q_n)_n$ be a DAC with a finite second moment and that satisfies a second-moment deviation inequality. Then
	 	\[
	 	\lim_{M\to \infty}\limsup_{n\to \infty}\frac{1}{n}V\Big(Q_n-\sum_{j=0}^{\lceil n/2^M \rceil-1}Q_{2^M}\circ \theta_{j2^M}\Big)=0.
	 	\]
	 \end{lem}
	 \begin{proof}
	 	This should be done more carefully.
	 \end{proof}







\section{General bounds for the defect of the TSP}
In this section we explain the key inequality for the TSP along the trajectory of the random walk. We will use this inequality to bound the norm of defective adapted cocycles in the next sections.

Let $(X,d)$ be a proper geodesic metric space. Recall that for the points $P, Q \in X$ and a finite set of points $L$, we denote by the $TSP(P,L,Q)$ the length of the shortest path in $X$ which starts at $P$, ends at $Q$, and visits every point in $L$. Let $\alpha$ be some solution to the $TSP(P,L,Q)$ and let us list the points of $L=\{l_1,\ldots,l_k\}$ in the order of their fist appearances along $\alpha$ $L=(l_{\pi(1)},\ldots,l_{\pi(k)})$. Then we have the following equality $$TSP(P,L,Q)
=d(P,l_{\pi(1)})+\sum_{i=1}^{k-1}d(l_{\pi(i)},l_{\pi(i+1)})+d(l_{\pi(k)},Q)$$ and the permutation $\pi \in \mathrm{Sym}(k) $ determines $\alpha$ uniquely up to the choice of the geodesic segments connecting $P$ with $l_{\pi(1)}$, $Q$  with $l_{\pi(k)}$ and $l_{\pi(i)}$ with $l_{\pi(i+1)}$ for $i=1, \ldots, k-1$. For any solution $\alpha$ of the $TSP(P,L,Q)$, we will refer to the points in $\{P,Q\}\cup L$ as the \textit{nodes} of $\alpha$. In the proof of the next lemma we will view $\alpha$ as a homeomorphism from some segment into $X$, and whenever we talk about a subpath of $\alpha$ we mean the restriction of this homeomorhpism to a closed subsegment.
\red{Here we should change the symmetric difference by a union}
\begin{lem}\label{comb_argument}
Let $P,Q,R$ be three distinct points in the metric space $X$. Pick two finite sets $L_1, L_2 \subset X$.  We will call the points in  $L_1 \cup L_2 \cup \{P,Q,R\}$ the marked points. Let $I,C,T \subseteq X$ be three bounded sets such that the following conditions hold.
\begin{enumerate}
    \item $P \in I$, $Q \in C$ and $R \in T$.
    \item $ I \cap T =\emptyset $. 
    \item $ L_1 \subseteq I \cup C$ and 
    $ L_2 \subseteq C \cup T$.
    
\end{enumerate}

Moreover, assume that there are two compact sets $B_1, B_2 \subseteq X$  such that the following conditions hold.
\begin{enumerate}
\item $B_1 \cap T= B_2 \cap I= \emptyset$.
 \item Any geodesic segment joining a marked point in $I$ with a marked point in $C \cup T$ intersects $B_1$, and any geodesic segment joining a marked point in  $T$ with a marked point in $I \cup C$ intersects $B_2$.
 \item If $x$ is any marked point in $I$ and $y$ is any marked point in $C$, then for any geodesic segment $\gamma$ connecting $x$ and $y$ the closest to $x$ point in  $\gamma \cap (B_1 \cup B_2)$ is in $B_1$ and the closest to $y$ point in $\gamma \cap (B_1 \cup B_2)$ is in $B_2$.
 \item If $D_1$ is the maximum of diameters of $B_1$ and $B_2$, then $d(B_1,B_2) > 2D_1$.
\end{enumerate}
  Let $D_2$ denote the diameter of $B_1 \cup C \cup B_2$ and let $N$ be the number of marked points in $C$. Then the following inequality holds.
\begin{align*}
    & 0 \leq TSP(P, L_1, Q)+TSP(Q, L_2, R) - TSP(P, L_1 \Delta L_2, R) \leq N(12D_1+2D_2)
\end{align*}

\end{lem}

\begin{proof}
 Since the concatenation of any solution of $TSP(P, L_1, Q)$ with any solution of $TSP(Q, L_2, R)$ at $Q$ produces a path that starts in $P$, ends at $R$, and visits every point in $L_1 \Delta L_2$, the first inequality follows.

 
 In order to prove the second part of the inequality we will show that any solution $\alpha$ of the $TSP(P, L_1 \Delta L_2, R) $ contains two non-overlapping parts $\alpha_I$ and $\alpha_T$, such that $\alpha_I$ is close to the solution of $TSP(P, L_1, Q)$ and $\alpha_T$ is close to the solution of $TSP(Q, L_2, R)$.
 
 Let $\alpha$ be any path from $P$ to $R$ realizing the solution to $TSP(P, L_1 \Delta L_2, R)$. We define \textit{the trace} of $\alpha$ in $I \cup B_1$ denoted by $\alpha_I$ as follows. Let $\pi$ be the permutation of $L_1 \Delta L_2 =\{x_1,\ldots,x_k\}$ that defines $\alpha$ and let $s_i$ be a geodesic segment of $\alpha$ joining $x_{\pi(i)}$ and $x_{\pi(i+1)}$. Then, if both endpoints of $s_i$ belong to $I$ we include $s_i$ into the trace $\alpha_I$. If exactly one of the endpoints of $s_i$, let us denote it  by $x$, is in $I$, then we can find the closest to $x$ intersection of $s_i$ with $B_1$, denoted by $y$, and add the part of $s_i$ between $x$ and $y$ to the trace $\alpha_I$. Otherwise, no new points from $s_i$ are added to the trace.


It is easy to see that $\alpha_I$ is a union of maximal subpaths $p_1,\ldots p_t$ of $\alpha$, such that for any $i=1,\ldots, t$ the subpath $p_t$ has both of its endpoints in $B_1 \cup \{P\}$.  Now, we are going to show that one can add a collection of geodesic segments of total length at most $N(6D_1+D_2)$ to $\alpha_I$  to get a path that starts at $P$, ends at $Q$, and visits every point in $L_1$. 

Since the sets $I,C,T$ satisfy the conditions (1)-(3) from the statement of the lemma, we have $L_1 \cap (I \setminus C) =(L_1 \Delta L_2) \cap (I \setminus C)$, so $\alpha_I$ already contains every point in $L_1 \cap (I \setminus C)$. Therefore, if $\beta$ is any path that starts at $P$, contains $\alpha_I$, and ends at a point in $B_1$, then one can extend $\beta$ to a path that visits every point of $L_1$ and ends at $Q$  as follows.  First, we connect the points in $L_1 \cap C$ by geodesic segments in arbitrary order, and then one of the endpoints of the resulting path is connected with the endpoint of $\beta$ that lies in $B_1$, while  the other end of this path is connected to $Q$. Notice that the total length of the geodesic segments that we add to $\beta$ in this procedure will not exceed $ND_2$.

Next we will construct a suitable path $\beta$ by joining the subpaths $p_1,\ldots, p_t$ of $\alpha_I$ by at most $6N$ geodesic segments, with each segment having the length at most $D_1$. We need  the following fact.
\begin{claim}
 Let $p_1,\ldots,p_t$ be the list of the subpaths of $\alpha_I$ defined above. Then $t \leq 6N$.
\end{claim}

\begin{proof}
  Let $\pi$ be the permutation defining $\alpha$ and let $s_1,...s_{k+1}$ be the corresponding geodesic segments. Notice that by definition each of these segments corresponds to the first visit of $\alpha$ to a new point from $ L_1 \Delta L_2$.  

  We will trace the paths $p_1, \ldots, p_t$ as we travel along $\alpha$. By definition, each path $p_i$, except for $p_1$,  starts and ends with subsegments of uniquely determined geodesic segments $s'_i$ and $s_i$ of $\alpha$ which connect marked points with points in $B_1$. The other endpoint of $s_i$ must be a new marked point that belongs either to $C \setminus T$ or to $T$. If this marked endpoint is in $T$ we will call $s_i$ a \textit{leap}, and if it is in $C \setminus T$, we will call $s_i$ a \textit{step}. It is easy to see that the number of \textit{steps} can not exceed the number of marked points in $C \setminus T$ and this number is less or equal to $N$. Therefore, it suffices to show that the number of leaps can not exceed $5N$. 
  
  We will prove even stronger statement, namely, that the number of the geodesic segments of $\alpha$ joining marked vertices in $I$ with marked vertices in $T$ can not exceed $5N$.  For the following combinatorial argument, it will be convenient to introduce the coding of the nodes of $\alpha$ by symbols $\mathcal{I},\mathcal{C},\mathcal{T}$. Naturally, a point is coded by $\mathcal{I}$ if it belongs to $I$, by $\mathcal{C}$ if it belongs to $C \setminus(I \cup T)$ and by $\mathcal{T}$, if the point is in $T$. Thus, if $\alpha$ has $k+2$ nodes including $P$ and $R$, the coding will produce a word $ \omega$ of length $k+2$ starting with letter $\mathcal{I}$ and ending with $\mathcal{T}$ which contains no more than $N$ letters $\mathcal{C}$.  Moreover, every geodesic segment of $\alpha$ joining a node in $I$ with a node in $T$ corresponds to a unique subword $\mathcal{IT}$ or $\mathcal{TI}$ in $\omega$. We claim that if the number of subwords $\mathcal{IT}$ or $\mathcal{TI}$ in $\omega$ is at least $5N$, then $\alpha$ could be replaced by a strictly shorter path contradicting the definition of the solution to the TSP. Indeed, assume that $\omega$  contains at least $5N$ subwords of the form $\mathcal{IT}$ or $\mathcal{TI}$, then since it contains only $N$ letters $\mathcal{C}$, there exists a subword of $\omega$ which contains $4$ consecutive subwords of the form $\mathcal{IT}$ or $\mathcal{TI}$ and does not contain letter $\mathcal{C}$. Hence this word contains a subword of the form  $\mathcal{I}\mathcal{T}^a\mathcal{I}^b\mathcal{T}$ where $a,b \geq 1$. In geometric terms this means that $\alpha$ contains a path $\gamma$ of the following form: 
  \begin{enumerate}
      \item $\gamma$ contains three geodesic segments of $\alpha$ $P_1Q_1, Q_2P_2$, and $P_3Q_3$ such that $P_1, P_2, P_3 \in I$ and $Q_1,Q_2, Q_3 \in T$, and $\gamma$ starts with the segment $P_1Q_1$ and ends with the segment $P_3Q_3$.
      \item all nodes of $\gamma$ between $Q_1$ and $Q_2$ are contained in $T$, while all of its nodes visited  between $P_2$ and $P_3$ belong to $I$.
  \end{enumerate}

Notice that conditions (1)-(3) on $B_1$ and $B_2$ imply that each of the segments $P_1Q_1$ and $P_2Q_2$ intersects both $B_1$ and $B_2$, and the length of each of these segments is at least $2D_1$.  

Now we are going to run a "surgery" procedure that produces a suitable shortening $\gamma'$ of $\gamma$.
Let $x_1$ and $x_2$ be closest to $P_1$ and to $P_2$, respectively, intersections of $P_1Q_1$ and of $Q_2P_2$ with $B_1$, and for $i=1,2,3$ we define $y_i$ as the closest to $Q_i$ intersection of $P_iQ_i$ with $B_2$. Then $\gamma'$ is constructed as follows. It starts at $P_1$ and tracks $P_1Q_1$ until it reaches $x_1$, then $\gamma'$ moves from $x_1$ to $x_2$ and follows $\gamma$ until it reaches $y_3$ visiting all of the nodes between $P_2$ and $P_3$ in the process. After $\gamma'$ reaches $y_3$ it moves to $y_1$ and follows $\gamma$ until it reaches $y_2$ visiting all of the nodes between $Q_1$ and $Q_2$ in the process. Finally, from $y_2$ $\gamma'$ moves to $y_3$ and follows $\gamma$ to $Q_3$.

It is easy to see that $\gamma'$ starts at $P_1$, ends at $Q_3$, and visits all of the nodes of $\gamma$.  Moreover, $\gamma'$ is obtained by removing segments $x_1y_1$ and $x_2y_2$, each of length greater than $2D_1$, from $\gamma$, and adding the segments $x_1x_2$, $y_3y_1$ and $y_2y_3$. Since $D_1$ is the maximum of the diameters of $B_1$ and $B_2$, the total length of the segments added does not exceed $3D_1$, so $\gamma'$ is indeed strictly shorter than $\gamma$.

Finally, if one replaces $\gamma$ in $\alpha$ by $\gamma'$, the resulting path still starts at $P$, ends at $R$, and visits every point in $L_1 \Delta L_2$, but is shorter than $\alpha$, and this contradicts the assumption that $\alpha$ realizes the solution to the $TSP(P, L_1 \Delta L_2, R)$.
Therefore, the number of leaps is also bounded by $5N$, and  $t \leq 6N$.
\end{proof}
Now the construction of $\beta$ is completed as follows. For $i=1,\ldots,t-1$ connect the ending point of $p_i$ in $B_1$ with the starting point of $p_{i+1}$ in $B_1$ by a geodesic segment, by definition of $D_1$ such a segment of would have length at most $D_1$. Since $t \leq 6N$, we have added the segments of total length at most $6ND$ to $\alpha_I$, and it is easy to see that the resulting path starts at $P$, contains $\alpha_I$, and ends at a point in $B_1$.

Therefore, we can conclude that the length $l(\alpha_I)$ satisfies the inequality $$ l(\alpha_I) \geq TSP(P, L_1, Q) -N(6D_1+D_2).$$

The trace $\alpha_T$ is defined similarly, and a similar argument shows that $$ l(\alpha_T) \geq TSP(P, L_2, Q) -N(6D_1+D_2).$$

It is easy to see that no node of $\alpha$ appears in  both $\alpha_I$ and $\alpha_T$, so they have no subpath of $\alpha$ in common, and therefore, we have 
\begin{align*}
   &TSP(P, L_1, Q)+TSP(Q, L_2, R)-N(12D_1+2D_2) \leq l(\alpha_I)+l(\alpha_T)  \leq l(\alpha)
\end{align*}
 This completes the proof of the second inequality.


 
\end{proof}
\section{Acylindrically hyperbolic groups}

In this section we will use lemma \ref{comb_argument} to obtain the upper bounds on the moments of the defective adapted cocycles when the base group $H$ is acylindrically hyperbolic. 

We remind that our goal is to prove an inequality of the following kind: there is a polynomial with positive coefficients $p$ (probably power 5 or 6) such that for any $n.m\ge 1$ we have
\[
\E(|\|\Psi_{n,m}|^2)\le p(\log(n)+\log(m)).
\]

With this in mind, for finitely supported random walk, then we can restrict to supposing $n>C\log(n+m)$ for some big constant $C$. Indeed, if $n\le C\log(n+m)$, then  $\Psi_{n,m}\le C n$ (it is after all a difference of the word lengths).



The following proposition is a straightforward corollary of Theorem 9.1 and Theorem 10.7 in \cite{MathieuSisto2020}

\begin{prop}\label{prop: tracking}
Let $H$ be a finitely generated acylindrically hyperbolic group and let $\mu_H$ be a symmetric non-elementary probability measure on $H$ with finite exponential moment.  Choose arbitrary finite symmetric generating set of $H$ and let $d_H$ be the corresponding  word metric on $H$.  Finally, we denote by $Z_n$ the Then the following statements hold.


\begin{enumerate}
    \item There exists a constant $K$ such that  for any $n \geq 1$
    \begin{align*}
      \mathbb{P}^{\mu_H} \left( d_H(Z_n,e_H) \leq n/K \right) \leq Ke^{-n/K}
    \end{align*}
    \item [uniform geodesic tracking] There is a constant $C$ such that for any $n \geq 1$, for each pair $(i,j)$ with $1 \leq i < j \leq n $  and each geodesic segment $\alpha$ joining $Z_i$ and $Z_j$ we have
    \begin{align*}
      \mathbb{P}^{\mu_H} \left( \max_{ i\leq k \leq j} d_H(Z_k,\alpha) \geq C \log n \right) \leq C/n^4
    \end{align*}
\end{enumerate}
\end{prop}

\begin{proof}
The first statement immediately follows from Theorem 9.1, Remark 10.2 and statement 1 in Proposition 10.3 in \cite{MathieuSisto2020} 

The second statement follows from Theorem 10.7 and Remark 10.2 in \cite{MathieuSisto2020} combined with the union bound.
\end{proof}
\red{We should decide whether to replace the first item with union bound.}

\begin{lem}\label{lem: uniform progress}
	Let $K_0$ be the constant of Proposition \ref{prop: tracking}. Then there is a constant $K_1$ such that with probability $1-\frac{1}{(n+m)^2}$ we have the following. As soon as $|i-j|\ge K_1\log(n+m)$, for $i,j\in \{0,\ldots n+m\}$, we have $d_H(Z_i,Z_j)\ge \frac{|i-j|}{K_0}$.
\end{lem}
\begin{proof}
	Union bound using Proposition \ref{prop: tracking}.
\end{proof}

Our aim is to reduce the situation to a deterministic setting that occurs with high probability. Then, we apply the combinatorial argument, and afterwards we estimate the moments associated with the constants that appear in the combinatorial lemma, using that the deterministic setting occurs with high probability.

From now on consider $n,m\ge 1$ such that $n,m \geq C_0\log(n+m)$.

From now on, using the uniform geodesic tracking from Proposition \ref{prop: tracking}, we will assume that  there is a constant $C_1$ for each pair $(i,j)$ with $1 \leq i < j \leq n+m $  and each geodesic segment $\alpha$ joining $Z_i$ and $Z_j$ we have

\[
\max_{ i\leq k \leq j} d_H(Z_k,\alpha) \leq C_1 \log (n+m),
\]
with probability at least $1-C_1\frac{1}{(n+m)^4}$.

From now on, let us denote
$$
W=C_1\log(n+m).
$$



\subsection{Definitions of $\mathfrak{I}$, $\mathfrak{M}$  and $\mathfrak{T}$}
We will first define the sets $\mathfrak{I}$, $\mathfrak{M}$ and $\mathfrak{T}$, and then explain why they have the properties that we need.

Now we fix a constant $C_2$ much larger than $C_1$. 

We take $R=d_H(e_H,Z_m)$. Let us fix a geodesic $\alpha$ from $e_H$ to $Z_{n+m}$.  Consider the neighborhood of $\alpha$ of radius $W$:

$$
N_{W}(\alpha)=\{g\in H\mid d_H(\alpha, g)\le W\}.
$$

 We define 

\[
\mathfrak{I}\coloneqq \{g\in H\mid d_H(g,e_H)\le R-C_2\log(n+m)\}\cap N_W(\alpha).
\]

\[
\mathfrak{T}\coloneqq \{g\in H\mid d_H(g,e_H)\ge R+C_2\log(n+m)\}\cap N_W(\alpha).
\]

\[
\mathfrak{M}\coloneqq \{g\in H\mid R-C_2\log(n+m) \le d_H(g,e_H)\le R+C_2\log(n+m)\}\cap N_W(\alpha).
\]


Matthieu-Sisto, with our choice of $n$ and $m$, will guarantee that $\mathfrak{I}$ and $\mathfrak{T}$ will be non-empty.

By definition we have $Z_0=e_H\in \mathfrak{I}$ and $Z_m\in \mathfrak{M}$. We will prove that with high probability, the trajectory of the random walk between times $1$ and $m$ does not enter $\mathfrak{T}$, trajectory between $m$ to $m+n$ does not enter $\mathfrak{I}$. In particular, $Z_{m+n}\in \mathfrak{T}$ with high probability.

From the definition, we have $\mathfrak{I}\cap \mathfrak{T}=\varnothing$. For us, $L_1=\{Z_0,Z_1,\ldots, Z_m\}$ and $L_2=\{Z_{m+1},\ldots, Z_{m+n}\}.$

Let us now define $B_1$ and $B_2$.

\[
B_1\coloneqq \{g\in H\mid d_H(g,e_H)=R-C_2\log(n+m)\}\cap N_{4W}(\alpha).
\]

\[
B_2\coloneqq \{g\in H\mid d_H(g,e_H)=R+C_2\log(n+m)\}\cap N_{4W}(\alpha).
\]

By definition we have $B_1\cap \mathfrak{I}=B_2\cap \mathfrak{T}=\varnothing$. This verifies the first condition. The third condition follows from the definition. The fourth condition will follow from the fact that we chose $C_2>>C_1$. The second condition will be the most technically challenging to verify. It will follow from our assumption of uniform geodesic tracking.

Now we will prove that the sets we have defined together with the points $Z_0$, $Z_m$ and $Z_{n+m}$ verify the conditions of the combinatorial lemma.



\begin{lem}
	If we choose $C_2>>C_1$ large enough, under the assumptions of uniform geodesic tracking and Lemma \ref{lem: uniform progress}, we can guarantee that $\{Z_0,Z_1,\ldots ,Z_m\}\cap \mathfrak{T}=\{Z_m,Z_{m+1},\ldots ,Z_{m+n}\}\cap \mathfrak{I}=\varnothing$.
\end{lem}
\begin{proof}
	Otherwise, you can find points $Z_i$, $Z_j$ such that $d_H(Z_i,Z_j)\le  \frac{|i-j|}{K_0}$ with $|i-j|\ge 3 C_1 K_1\log(n+m)$.
	
	Now, if we choose $C_2> 3C_1K_1\cdot (step\ length)$ the random walk cannot possibly enter $\mathfrak{I}$ after moment $m$. This proves that $\{Z_m,Z_{m+1},\ldots ,Z_{m+n}\}\cap \mathfrak{I}=\varnothing$. Then a similar reasoning shows $\{Z_0,Z_1,\ldots ,Z_m\}\cap \mathfrak{T}=\varnothing$.
	
\end{proof}

\begin{lem} Any geodesic segment joining a point of the trajectory of the random walk $(Z_k)_k$ inside $\mathfrak{I}$ with a marked point in $\mathfrak{M} \cup \mathfrak{T}$ intersects $B_1$, and any geodesic segment joining a  point of the trajectory of the random walk $(Z_k)_k$ inside $\mathfrak{T}$ with a marked point in $\mathfrak{I}\cup \mathfrak{M}$ intersects $B_2$.	
\end{lem}
\begin{proof}
	We will provide the proof for $B_1$; for $B_2$ it is analogous.
	
	Suppose that we have $Z_i\in \mathfrak{I}$ and $Z_j\in \mathfrak{M}\cup \mathfrak{T}$. Then we can track trajectory of the random walk between $i$ and $j$ and join them with geodesic segments. This gives a path that we call $P_{i,j}$.
	
	Let us denote by $C$ the maximal jump of the random walk. Denote by $\gamma$ a geodesic joining $Z_i$ and $Z_j$. Then for any $y\in P_{i,j}$ is within distance at most $W+C$ from both $\alpha$ and $\gamma$ (this follows from uniform geodesic tracking).
	
	To prove our claim, it suffices to show that any point in $\gamma$ is within $4W$ distance of $\alpha$. Then the claim follows from continuity of distance (i.e. at some point the path crosses the appropriate sphere).
	
	Take any $y\in \gamma$. We will prove that there is a point in $\alpha$ at distance at most $4W$ from $y$. Let us split $\gamma$ into two closed segments $[Z_i,y]$ and $[y,Z_j]$. Then any point in $P_{i,j}$ is at distance at most $W+C$ from at least one of these subsegments. By connectivity, we can find a point $z\in P_{i,j}$ such that $z$ is at distance at most $W+C$ from both segments. Let us denote $\pi_1(z)\in [Z_i,y]$ and $\pi_2(z)\in [y,Z_j]$ such that $d_H(z,\pi_1(z))\le W+C$ and $d_H(z,\pi_2(z))\le W+C$. The subpath of $\gamma$ that connects $\pi_1(z)$ and $\pi_2(z)$ will be geodesic and contain $y$. Then, by triangular inequality, the length of this geodesic subpath is at most $2W+2C$. suppose without losing generality that $d_H(y,\pi_1(z))\le W+C$. Then the path that connects $y$ to $\pi_1(z)$ to $z$ and then to the projection of $z$ on $\alpha$ has length at most $3W+3C$, and we can assume that $W$ is larger than $3C$. Hence we conclude the upper bound of $4W$.
\end{proof}


\subsection{Estimates for the moments of $N$, $D_1$ and $D_2$}

From what we have done so far we have $D_1\le 16 W$ and $D_2\le 16W +C_2\log(n+m)$. Now we will bound $N$.

For $N$ we have that it is bounded by the diameter of $\mathfrak{M}$ times $K_0$. This follows from uniform progress of the random walk. This gives the upper bound $N\le K_0 (16W+2C+2\log(n+m))+1$.


\section{Hyperbolic groups}

We do as above, following similarly to the combinatorial proof I wrote with Kunal using pivots. We need to phrase it, as above, in terms of the combinatorial lemma.
\section{CLT for the lamplighter over a hyperbolic group (using pivots)}
%%% Proof using pivots

%\begin{thm}
%	Let $A$ be a non-trivial group and $H$ a non-elementary hyperbolic group. Consider a probability measure $\mu$ on $A\wr H$ such that
%	\begin{enumerate}
%		\item $\supp(\mu_H)$ is non-elementary, and
%		\item $\mu_H$ has a finite second moment,
%		\item $\mu=\mu_A*\mu_H$ (so the rw modifies the lamp in the current position and not anywhere else).
%	\end{enumerate}
%	Denote $\{w_n\}_{n\ge 0}$ the $\mu$-random walk on $A\wr H$. Then $\{|w_n|\}$ satisfies the CLT.
%\end{thm}
%
%Here for any $g=(f,x)\in A\wr B$ the word length $|g|$ is
%\[
%|g|= \sum_{b\in\supp(f)}|f(b)|_{A} + \mathrm{TSP}(e_H,x,\supp(f)).
%\]
%
%We will use the following general criterion of Mathieu-Sisto.

%\begin{thm}[\cite{MathieuSisto2020}] \label{thm:generalCLT}
%	Suppose that $Q_n$ is a defective adapted cocycle with defect
%	\[
%	\Phi_{m,n} := Q_n - (Q_m + w_m Q_{m-n})
%	\]
%	such that  there exists $C>0$  \[
%	\E[|\Phi_{m,n}|^2] \le C
%	\]
%	uniformly on $m$ and $n$. Then the CLT holds for $Q_n$.
%\end{thm}
\subsection{Pivots}
Let us consider a non-elementary hyperbolic group $H$, and let us fix a finite generating set $S_H$. Let $\delta\ge 0$ be the hyperbolicity constant of $\cay{H}{S_H}$, and let us denote by $d_H:H\times H\to \mathbb{Z}_{\ge 0}$ the word metric on $H$ with respect to $S_H$.

\begin{defin}
	Given a path $\gamma=(\gamma_1,\gamma_2,\ldots,\gamma_k)$ on $\cay{H}{S_H}$ and $g\in H$, let $\pi_{\gamma}(g)$ be the set of elements of $\gamma$ that minimize the word metric to $g$. That is, we define
	\begin{equation}
		\pi_{\gamma}(g)\coloneqq \left\{ \gamma_i \in \gamma \mid  d_S(\gamma_i,g)\le d_S(\gamma_j,g) \text{ for all }j=1,\ldots,k \right \}.
	\end{equation}
\end{defin}

We now introduce the definition of pivots that we will use in the proof of Theorem \ref{thm:lamplighterCLT}. We refer to \cite[Section 4A]{Gouezel2022} for details.

\begin{defin}
	Let $C,D>0$, $L\ge 20C+100\delta+1$, and $N\in \Z_{\ge 1}$. Let $\mathbf{w}=\{w_n\}_{n\ge 0} \in (A\wr H)^{\mathbb{N}}$ be a sample path of the $\mu$-random walk on $A\wr H$. Denote by $w_n^{H}$ the projection to $H$ of $w_n$, for each $n\ge 0$. A time instant $m\ge 1$ is a \emph{$(C,D,L,N)$-pivot} for the sample path $\mathbf{w}$ if the following three conditions hold. 
	\begin{enumerate}
		\item $d_{H}\left(w^{H}_m,w^{H}_{m+N}\right)\ge L$.
	\end{enumerate}

 Let $\gamma$ be an arbitrary geodesic path in $\cay{H}{S_H}$ that connects $w^{H}_m$ to $w^{H}_{m+N}.$ Then 
 
	\begin{enumerate}\setcounter{enumi}{1}
	\item
	\[ 
	d_H\left( \pi_{\gamma}\left(w^{H}_k\right),w^{H}_m \right)\le C \text{, for all }k\in\{0,1,\ldots, m\},
	\]
	\item for all $m\le k\le m+N$ we have $d_H\left(w^{H}_k, \gamma \right)\le D$, and
	\item for all $k\ge m+N$, we have $d_H\left( \pi_{\gamma}\left(w_k^H \right), w^{H}_{m+N} \right)\le C$.
\end{enumerate}
\end{defin}

The following lemma will be our main tool.

\begin{lem}
	For any $C, D, \delta>0$, and any $L\ge 20C+100\delta+1$, there exists $N,R>0$ large such that 
	\[
	\sup_{i\ge 1}\P\left(\exists m\in [i,i+k]\text{ such that }m\text{ is an } (C,D,L,N)\text{-pivot}\right)\ge 1-Re^{-k/R}.
	\]
\end{lem}
\begin{proof}[Sketch of proof]
	This is proven in proposition 4.11 in Gouezel's paper, where for Gouezel's definition of pivots, conditions 1,2, and 4 are met. To see why Gouezel's proof implies the lemma we state, we observe that for Gouezel's definition of pivots, the increments $ w _{n} ^{-1} w _{n+N} $ are drawn from some explicit finite set of isometries $ S \subset H $ that receive positive support from $ \mu_{H} ^{N} $. For this finite set of isometries, we can pick some $ D>0 $ large enough so that $ \mu ^{N} $ gives positive mass to each of the sets $ \pi _{H} ^{-1} (s) = \{(\varphi, s), \text{supp} \varphi \subset B _{D} ([e,s])\}$. Then tracing through the rest of Gouezel's proof we have the estimate required. 
\end{proof}


\subsection{TSP structure along pivots}

\begin{prop}\label{prop: structure along pivots}
	Suppose that we are looking at a sample path $\{w_n\}_{n\ge 0}$ and that we have a pivoting time $m$. Then the group element $w_n=(f_n,x_n)$ satisfies the following. The support of $f_n$ can be decomposed as a disjoint union
	\[\supp{f_n}=P^{\prime}_1\cup P^{\prime}_2\cup P^{\prime}_3,\]
	that satisfies the following properties. Let us denote $P_1=P^{\prime}_1\cup\{w^{H}_{m+u}\}$, $P_2=P^{\prime}_2\cup \{w^{H}_{m+u}, w^{H}_{m+u+N}\}$ and  $P_3=P^{\prime}_3\cup \{w^{H}_{m+u+N}\}$. Let  $\gamma$ be an arbitrary geodesic path from $w_m^{H}$ to $w^{H}_{m+N}$ on $\cay{H}{S_H}$. Then we have
	\begin{enumerate}
		\item for all $g\in P_1$, we have $d_H\left( \pi_{\gamma}(g) ,w_m^H  \right)\le C$,
		\item for all $g\in P_2$ we have $d_H(g,\gamma)\le D$, and
		\item for all $g\in P_3$, we have $d_D\left( \pi_{\gamma}(g),w^H_{m+N} \right)\le C$.
	\end{enumerate}
\end{prop}


\begin{defin}
	 Let $g=(f,x)\in A\wr H$ and suppose that $\supp{f}=P_1\cup P_2\cup P_3$. Let $\eta$ be a path on $\cay{A\wr H}{S_{\mathrm{sws}}}$ that realizes $|g|_{S_{\mathrm{sws}}}$. We define the associated \emph{coding} of $\eta$ as the word $u$ in the alphabet $\{P_1,P_2,P_3\}$, such that $u_i=P_j$ if and only if at the $i$-th step of $\eta$, there is a lamp at a position in $P_j$ which was modified for the first time.
\end{defin}

We are going to abuse notation (in this draft) and not make a distinction between the elements visited by a path, and the coding in the alphabet $\{P_1,P_2,P_3\}$ associated with it.
\begin{defin}
	Given a path $\eta$, let us call a \emph{backtracking} a subpath of $\eta$ that is of the form $P_1P_2^{*}P_3^{\varepsilon}P_2^{*}P_1^{\varepsilon^{\prime}}P_2^{*}P_3$ for $\varepsilon, \varepsilon^{\prime}\ge 1$. Here the $*$ symbolizes $0$ or more occurrences.
\end{defin}

\begin{lem}\label{lem: groceries lemma p1p3p1p3}
	Let $\eta$ be a solution to the TSP for $|w_n|$. Then the coding of $\eta$ does not have a subword of the form $P_1P_3^{\varepsilon}P_1^{\varepsilon^{\prime }}P_3$, for $\varepsilon, \varepsilon^{\prime}\ge 1$.
\end{lem}
\begin{proof}
	Surgery, meaning that you glue together the excursions to $P_1$, and you glue together the excursions to $P_3$, and connect them with any path through $P_2$. This gives something even shorter than optimal since each gluing strictly reduces the length of the path.
\end{proof}

\begin{cor}\label{cor: number of backtrackings}
	Let $\eta$ be a solution to the TSP for $|w_n|$. Then the number of backtrackings of $\eta$ is at most $|P_2|$.
\end{cor}
\begin{proof}
	Every backtracking must contain at least one element of $P_2$. (Recall that the path only has an element in its coding if it has not been visited before).
\end{proof}

\begin{lem}
	Consider a sequence of points $\{w_n\}_n$ of $H$, that satisfies the decomposition of $\supp{f_n}$ given by the three conditions of Proposition \ref{prop: structure along pivots}.
	%, such that $\pi_{\gamma}(w_0)$ is within distance $C$ of the beginning of $\gamma$, and $\pi_{\gamma}(w_n)$  is within distance $C$ of the end of $\gamma$. Also we have that the length of $\gamma$ is at least $L$ (where $L$ is the large constant from the definition of pivots). (this is equivalent to satisfying the three properties above).
	
	Let $T$ be the length of a solution to $\mathrm{TSP}(w_0,w_n,\supp{f_n})=\mathrm{TSP}\left(w^H_0,w^H_n,P_1\cup P_2\cup P_3\right)=|w_n|_{A \wr H}$. Then there exists a path $\eta$ that starts at $w^H_0$, finishes at $w^H_n$ and visits all points in $\supp{f_n}$ such that $length(\eta)\le T+100 N (L+2D)$, and such that, in the coding of $\eta$, all the elements of $P_1$ appear before any of the elements of $P_3$.
	
%	The path $\eta$ induces a linear order of $\supp{f_n}$. What we require is that in the coding of $\eta$, all the elements of $P_1$ appear before any of $P_3$. (i.e. the coding does not have a subsequence of the form $P_3 P_1$).
\end{lem}
\begin{proof}
	
	Let us first consider $\eta_0$ the optimal solution to the TSP. 
	%\textbf{Claim: } Any optimal solution to the TSP never sees a sequence of the form $P_1P_3^{*}P_1^{*}P_3$.
	%\textbf{Proof: }Surgery, meaning that you glue together the excursions to $P_1$, and you glue together the excursions to $P_3$, and connect them with any path through $P_2$. This gives something even shorter than optimal since each gluing strictly reduces the length of the path.
	
	
	Lemma \ref{lem: groceries lemma p1p3p1p3} implies Corollary \ref{cor: number of backtrackings} that the total number of backtrackings is the size of $P_2$.
	
	Finally, the argument goes as follows: first do all excursions of $\eta_0$ on $P_1$, then visit all elements in $P_2$, and then do all excursion in $P_3$. In total we added at most $2D\times$(number of backtrackings)+(length of solution of TSP in P2 that visits all elements in P2). And the number of backtrackings is at most $|P_2|$ by the previous claim.
	
	%Start with the optimal solution to the TSP. This path will cross through $N_D(\gamma)$ a bounded number of times (at most the size of $P_2$). This implies that we can modify it (do surgery  near the starting and finishing points of $\gamma$) so that we obtain path that crosses through $N_D(\gamma)$ only once, while increasing length at each surgery a bounded number, and we do a bounded number of surgeries.
\end{proof}
% Below are some suggestions that could simplify the proof of the TSP part. They are based on the argument we had in the solution without pivots. It's great if you already have a complete proof of lemma 6, but if it truns out to be more time consuming that expected, I hope these suggestions could help.
% It may be convenient to slightly modify the construction to make sure that every geodesic segment joining a point in P_1 with a point in P_2 or in P_3 intersects some set $B_1$ of small (relative to L) diameter that lies close to $w_m$ and "separates" P_1 from P_2 and P_3. The condition on projections ensures that this holds for geodesic segments from $P_1$ to $P_3$, but we may need some additional assumptions to make sure that this holds for geodesic segments from P_1 to P_2.  Of course, we would also like to find a similar set B_2 for P_3.

% Kunal: thanks for the helpful suggestions! I agree that it is important that every geodesic segment joining a point in P_1 with a point in P_3 intersects a bounded neighbourhood of something 'separating' P_1 from P_2 and P_3. This is the role played by the geodesic \gamma in our arguments. At the pivotal time, we pass through some long geodesic such that (1) everything in P_1 projects near the beginning of gamma, (2) everything in P_2 is within a bounded neighbourhood of gamma, and (3) everything in P_3 projects near the end of gamma. By elementary hyperbolic geometry, this implies any geodesic connecting P_1 to P_2 or P_3 (or P_2 to P_3) must come within a bounded neighbourhood of gamma. In fact, this neighourhood of gamma serves the role of B_1 and B_2 simultaneously.

%These modifications would allow us to get an easy upper bound on the total length of the segments added in the "surgery". The fact that B_1 and B_2 are far apart would be helful in bounding the number of the corssings of the central region.
% It is also sufficient to bound the number of crossings of N_D as a constant-times- (size of P_2), we had an argument with 4*(size of P_2). To get this bound one can show that the TSP solution can not contain four consecutive crossings of the central region that do not add any point in P_2 to the TPS (otherwise, one can remove the long corossings and connect their intersections with B_1 and B_2 within these regions in appropriate order to get a shorter curve).

%Kunal: I absolutely agree, it seems like Eduardo and I had the same argument in mind.

In other words, we are saying that trying to solve the problem by first visiting all elements of $P_1$, and then visiting all elements of $P_3$, and crossing the middle section only once, is at a bounded length of being optimal.

\begin{lem}\label{lem: small values mn} For any $ N \in \mathbb{N} $ there exists some $ C > 0$ such that the following holds. set $ m \in \mathbb{N} $ be an integer and let $ U $ be the waiting time until the first pivot after time $ m $. Then we have
	$$\sup_{m\ge 1}\E\left[\left( \sum_{i=m}^{m+U+N}|g_i| \right)^2\right]\le C.$$
\end{lem}
\begin{proof}
	\textcolor{blue} { I'll go into the construction of pivots and explain why this is true. Maybe there's a simpler reasoning using only the exponential estimates on $ U $}.
	
	Recall that the way Gouezel constructs pivots is as follows: if we let $ S \subset H$ be our finite Schottky set, then we can decompose some convolution power $ \mu _{H} ^{N} $ as \[ \mu _{H} ^{N} = \alpha \mu _{S} + (1-\alpha)\nu \] for some positive $ \alpha >0$. Then we draw our increments as follows: let $ \{ \varepsilon _{i} \} _{i} $ be i.i.d. Bernoulli($ \alpha $) random variables. If $ \varepsilon _{i} = 1 $, we draw $ g' _{i} = s _{i} $ according to $ \mu _{S} $. Else we draw $g' _{i} = w _{i} $ according to $ \nu $. We observe that the sequence $ \{ g '_{1}...g' _{k} \} _{k} $ has the same distribution as $ \{ g _{1}...g _{k} \} _{k} $ for $ g _{i} \sim \mu _{H} ^{N} $.
	
	Now we denote the resampled random walk by $ g' _{1}...g' _{n} =w _{1}...w _{k _{1}} s _{1} w _{k _{1} +1} ...w _{k _{2}} s _{2} ... $, where the strings between $ s _{i}'s $ may be empty. Now each string $ w _{k _{i-1} +1}...w _{k _{i}} s _{i}  $ is distributed according to $ \nu ^{Z}* \mu _{S} $, where $ Z $ is a geometric random variable with parameter $ \alpha $. 
	
	Now Gouezel tells us that, conditional on any realization of the increments drawn from $ \nu $, the number of $ \mu _{S} $ increments $ \ell $ until we see a pivot has an exponential tail. This implies that
	\[ \mathbb{E} \left[\left(\sum_{i = m}^{m+U+N} |g _{i}| \right) ^{2} | \{ w _{i} \}_i \right] \leq \mathbb{E} \left[\left(L\ell + \sum_{i = 0}^{\ell-1} \sum_{k = K _{i}}^{K _{i+1} -1} |w _{i}|\right) ^{2} | \{ w _{i} \} _{i} \right] .\] 
	
	Now we can integrate over the possible values of $ w _{i} $ and use independence in order to conclude that 
	
	\[ \E\left[\left( \sum_{i=m}^{m+U+N}|g_i| \right)^2\right] \] is bounded uniformly over $ m $.
\end{proof}

\subsection{Proof of the CLT}

Let us define $\Phi_{n,m}=|w_n|-|w_m^{-1}w_n|-|w_m|$. Thanks to Theorem \ref{thm: MS general CLT with constant deviation ineq}, it suffices to show that $\sup_{m,n\ge 1}\E(|\Phi_{n,m}|^2)$ is bounded.

We will do the proof for finitely supported $\mu$.

We fix $m$ and $n$. Let $m+u$ be the first instant after $m$ that you see a pivot.

If $m+u+N>n$, then we use Lemma \ref{lem: small values mn} 
$$
\E |\Phi_{n,m}|^2\le \sup_{m\ge 1}\E\left[\left( \sum_{i=m}^{m+u}|g_i| \right)^2\right]\le C.
$$
Otherwise, $m+u+N\le n$ and we do the following.

%Find a pivot at time $m+u$ for some positive $u>0$ which satisfies $\P(u\ge j)\le Re^{-j/R}.$

\begin{enumerate}
	\item The three conditions at the beginning of this subsection are satisfies.
	\item Our objective is to get a good upper bound for $\Phi_{n,m}$ in the inequality
	\[|w_n|\ge |w_m|+|w_{m}^{-1}w_n|-\Phi_{n,m}. \]
	\item We first note that $\left| |w_m|-|w_{m+u}| \right|$ has a finite second moment. Indeed, this amount is controlled by the increments done during $u$ steps, and we know the distribution of how large $u$ can be. That is, we use Lemma \ref{lem: small values mn} to justify this.  The same is true for $\left| |w_m^{-1}w_n|-|w_{m+u}^{-1}w_n| \right|$. Again, this follows from a triangular inequality and Lemma \ref{lem: small values mn}.
	\item From this, we just need a good upper bound for $\Phi_{n,m}$ in the inequality
	\[|w_n|\ge |w_{m+u}|+|w_{m+u}^{-1}w_n|-\Phi_{n,m}. \]
	\item We note that $\left||w_{m+u+N}^{-1}w_n| -|w_{m+u}^{-1}w_n|\right|$ is a bounded constant (since it only depends on $N$), and in particular has a finite second moment.
	\item From this, we just need a good upper bound for $\Phi_{n,m}$ in the inequality
	\[|w_n|\ge |w_{m+u}|+|w_{m+u+N}^{-1}w_n|-\Phi_{n,m}. \]
	\item We look at the TSP between time $0$ and $n$, we use the path $\eta$ from the previous lemma to get a path which is near optimal and crosses only once the neighborhood of $\gamma$.
	\item From this path we obtain near-optimal paths from $|w_{m+u}|$ and for $|w_{m+u+N}^{-1}w_n|$, by doing surgery near the endpoints of $\gamma$ and possibly adding a constant bounded amount of length.
	
	Indeed, we first take the path from the starting point to the last visit to $P_1$, and we connect it to $w_{m+u}$. This is at most $Optimal+L+2D$. Similarly we look at the first time we enter $P_3$, and connect that to a path to $w_{m+u+N}$. This again adds at most $Optimal+L+2D$.
	\item From this, we directly apply \ref{thm: MS general CLT with constant deviation ineq}.
\end{enumerate}

%
In this section we explain a generalization of \cite[Theorems 4.1 \& 4.2]{MathieuSisto2020} (see Theorem \ref{thm: MS general CLT with constant deviation ineq})

\begin{thm} \label{thm: polylogarithmic MS CLT}
    Suppose that $Q_n$ is a defective adapted cocycle with defect
    \[
       \Phi_{m,n} := Q_n - (Q_m + Z_m Q_{m-n})
    \]
    and suppose that for some fixed polynomial $p$ and $N_0 \in \mathbb{N}$ we have that
    \[
       \E[|\Phi_{m,n}|^2] \le p(\log (n))
    \]
    whenever $m, n-m \ge N_0$.
    Then a CLT holds for $Q_n$.
\end{thm}
\red{Is this the formulation we want?}

Here are the places in the proof of Mathieu-Sisto's CLT where the deviation inequality is used:
\begin{enumerate}
    \item In Theorem 4.4 one obtains $V(Q_n)\le n\Big(p(4 \E(Q_1^2)+ 16\log(n))\Big)$
    \item Then in Lemma 4.5 it is cited a result of Hammersley \cite[Theorem 2]{Hammerslet1962}. This should be replaced by an inequality of the form $a_{n+m}\le a_n+a_m+bp(log(n+m))$.
    \begin{lem}
    	Let $\{a_n\}_{n\ge 0}$ be a sequence of non-negative real numbers. Suppose that there exists $b\ge 0$ and a polynomial $p:\R_{+}\to \R_{+}$ such that 
    	\[
    	a_{n+m}\le a_n+a_m+b\sqrt{p(log(a_n+a_m))}, \text{ for each }m,n\ge 0.
    	\]
    	Then the limit $\lim_{n\to \infty} \frac{a_n}{n}$ exists.
    \end{lem}
    \item The deviation inequality is used in multiple occasions during the proof of Lemma 4.6. One should do the appropriate modifications.
\end{enumerate}

%
\section{The CLT for lamplighter random walks on acylindrically hyperbolic groups}

\begin{thm} \label{thm:lamplighterCLT}
    Let $G = \Z/2\Z \wr H$ be 
    a wreath product over a Gromov-hyperbolic group $H$. Let $\tilde{Z}_n$
    be a random walk on $G$ with 
    step distribution which is finitely supported and \textcolor{red}{[what is the term for these types of generators? where everything is step or step+light]}. \textcolor{blue}{switch-walk-switch?}
    Then $d(o,\tilde{Z}_n)$ satisfies a CLT.
\end{thm}

\textcolor{red}{(Maybe adopt the convention that $\tilde{Z}_n$ is the random walk on the wreath product $G$, while $Z_n$ is its projection to the hyperbolic base group $H$. I think that the notation we've been using is that $R_n \subset H$ is the configuration of lights at time $n$, i.e.
$\tilde{Z}_n = (R_n, Z_n)$. Since this looks like the range of the random walk, maybe we want to change notation, but this is the notation I'm going to use as I write the lemmas we need for now.)}

For $R \subset H$, $x,y \in H$,
denote by $TSP(x,R,y)$ a solution to the traveling salesman problem, that is, the edge path in the Cayley graph of $H$ which starts at $x$, ends at $y$, visits each vertex in $R$, and has minimal length subject to this constraints.
Note that for our choice of generators for $G$, we have that $d((\emptyset,1),(R,x)) = |TSP(1,R,x)|$.

%\begin{thm}[{\cite{MathieuSisto2020}}] \label{thm:generalCLT}
%    Suppose that $Q_n$ is a defective adapted cocycle with defect
%    \[
%       \Phi_{m,n} := Q_n - (Q_m + Z_m Q_{m-n})
%    \]
%    and suppose that for some fixed polynomial $p$ and $N_0 \in \mathbb{N}$ we have that
%    \[
%       \E[|\Phi_{m,n}|^2] \le p(\log n)
%    \]
%    whenever $m, n-m \ge N_0$.
%    Then a CLT holds for $Q_n$.
%\end{thm}

\begin{lem}[Deterministic TSP comparison] \label{lem:deterministic}
   \textcolor{red}{When we write this out properly, we'll figure out what all the proper polynomials are, and the proper definition of the ``middle region'', etc.}
   There are fixed polynomials $p_1,p_2,p_3,p_4$ such that the following hold.
   Let $C_1, C_2,$ and $C_3$ be
   constants. 
   Let $(R_m,x_m), (R_n,x_n) \in G$
   and define $R_{m,n} \subset H$
   by 
   $(R_{m,n}, x_m^{-1} x_n) :=
   (R_m, x_m)^{-1} (R_n, x_n)$.
   Suppose that
   \begin{enumerate}
       \item $R_m \cup x_m R_{m,n} \subset \mathcal{N}(\xi, C_1 p_1(\log n))$,
       that is,
       every vertex of $R_m \cup x_m R_{m,n}$ is within distance 
       $C_1 p_1(\log n)$ of $\xi$, 
       where $\xi$ is a geodesic
       from $1$ to $x_n$.
       \item Setting $r := d(o,x_m)$,
       we have 
       $R_m \subset B(o,r+C_2 p_2(\log n))$ and
       $x_m R_{m,n} \subset
       B(o, r - C_2 p_2(\log n))^c$.
       \textcolor{red}{(Not exactly
       clear yet what will be the best way to define the ``middle/left/right'' of the path yet, but I'm using this definition at least for now since it probably will come most easily out of the Hoeffding speed concentration estimates.}
       \item Set
       \[
          M := B(o,r+C_2 p_2(\log n)) \cap B(o, r - C_2 p_2(\log n))^c.
       \]
       Then we have that
       \[
          |M \cap (R_m \cup x_m R_{n,m})| \le C_3 p_3(\log n).
       \]
   \end{enumerate}
   Then we have that
   \begin{align*}
      &|TSP(1,R_n,x_n)| \ge \\
      &|TSP(1,R_m,x_m)| +
      |TSP(1,R_{m,n}, x_m^{-1} x_n)|
      - (C_1 + C_2 + C_3)p_4(\log n).
   \end{align*}
   \textcolor{red}{Depending on the analysis, could maybe get some 
   different expression in the 
   $C_i$. If we have something
   of this form (or equivalently,
   a bound which is poly(log) times
   $\max(C_1, C_2, C_3)$),
   then to get the final step
   it will suffice to show the each
   $C_i$ is square-integrable (as 
   a random variable), but
   if other powers of the $C_i$
   become involved, might
   have to show stronger
   integrability.}
\end{lem}

\begin{proof}
The outline of the proof is as follows.
To simplify the notation, let $D=p_1(\log n)$. Notice that by quasi-convexity  there is an absolute constant $\delta$ such that for any two points $x,y \in N(\xi, D)$ any geodesic between $x$ and $y$ will be contained in $N(\xi,D+\delta)$.  



\begin{enumerate}
    \item  We will split $N(\xi, D)$ into three regions - the initial part $I$ close to the origin, the central region $C$ and the terminal part close to $x_n$.  More precisely,  we let $I:=N(\xi, D)\cap B(o, r - C_2 p_2(\log n)$, $C:= N(\xi,D) \cap B(o,r+C_2 p_2(\log n)) \cap B(o, r - C_2 p_2(\log n))^c$ and $T=N(\xi,D) \cap B(o, r + C_2 p_2(\log n))^c$. Polynomial $p_2$ will be determined later.
    
    \item  
    Pick any solution $S$ to  $TSP(1,R_n,x_n)$.  We are going to show that it is possible to add a collection of segments of total length bounded by a fixed polynomial in $\log_n$ to $\overline{I \cap S}$  to get a path starting at $1$, visiting every point in $R_m$ and ending at $x_m$.  Here  $\overline{I \cap S}$ stands for the union of $I \cap S$  together with all geodesic segments of $S$ that join two points in $R_n  \cap I $  but possibly go outside of $I$. Then by the definition of TSP this path would have the length bounded below by $|TSP(1,R_m,x_m)|$.  
    
    Similar argument will be applied to $\overline{S \cap T}$ and $|TSP(1,R_{m,n}, x_m^{-1} x_n)|$. Since $\overline{I \cap S}$ and $\overline{T \cap S}$ have no geodesic segments of $S$ in common,  we have  
    \begin{align*}
    &|TSP(1,R_n,x_n)|= |S| \ge |\overline{I \cap S}|+|\overline{T \cap S}| \ge \\ 
     &|TSP(1,R_m,x_m)| +
      |TSP(1,R_{m,n}, x_m^{-1} x_n)| -P(\log n)
    \end{align*}
\end{enumerate}
\end{proof}


Since $\Psi_{n,m}$ is bounded above by $3n$, we should be able to reduce the problem to deterministic case using  known deviation estimates and a simple union bound, so most likely we will not need the strongest versions of the next lemmas.

\begin{lem} \label{lem:boundC1}
    Define the random variable $\mathcal{C}_{1,m,n}$ to be the
    minimal value of $C_1$
    such that (1) in Lemma \ref{lem:deterministic} holds 
    for $\tilde{Z}_m = (R_m,x_m)$
    and $\tilde{Z}_n = (R_n,x_n)$,
    i.e.
    \[
       \mathcal{C}_{1,m,n} := 
       \inf \{ C_1 \ge 0 : 
       d(r,\xi) \le C_1 p_1(\log n) 
       \mbox{ for all } 
       r \in R_m \cup Z_m R_{n,m} \},
    \]
    where $\xi$ is a geodesic
    from $1$ to $x_n = Z_n$.
    Then
    \[
       \limsup_{m,n-m \to \infty} \E[ \mathcal{C}_{1,m,n}^2 ] < \infty.
    \]
    \textcolor{red}{The proof of this should be via a tail bound
    which should be contained in ``Tracking Rates of Random Walks'' by Sisto.}
\end{lem}
 Since $\Psi_{n,m}$ is bounded above by $3n$, we should be able to reduce the problem to deterministic case using  known deviation estimates and a simple union bound, so most likely we will not need the strongest versions of the next lemmas.
\begin{lem} \label{lem:boundC2}
    Define the random variable $\mathcal{C}_{2,m,n}$ to be the
    minimal value of $C_2$
    such that (2) in Lemma \ref{lem:deterministic} holds 
    for $\tilde{Z}_m = (R_m,x_m)$
    and $\tilde{Z}_n = (R_n,x_n)$.
    Then
    \[
       \limsup_{m,n-m \to \infty} \E[ \mathcal{C}_{2,m,n}^2 ] < \infty.
    \]
    \textcolor{red}{It seems likely enough that in the end the bound on $C_2$ will come simultaneously with the bound on either $C_1$ or $C_3$, since it
    should come from either tracking or speed. In fact, since $C_2$ is a parameter that we use to define the ``middle'' segment $M$, we will want to make it \emph{bigger} than the minimum possible. Maybe in the final argument we will just take 
    $C_2 = 6 C_1$ or $6 C_3$ or something like that and this lemma won't exist.}
\end{lem}

\begin{lem} \label{lem:boundC3}
    Define the random variable $\mathcal{C}_{3,m,n}$ to be the
    minimal value of $C_3$
    such that (3) in Lemma \ref{lem:deterministic} holds 
    for $\tilde{Z}_m = (R_m,x_m)$
    and $\tilde{Z}_n = (R_n,x_n)$.
    That is,
    \[
       \mathcal{C}_{3,m,n} :=
       \frac{|M \cap (R_m \cup x_m R_{m,n})|}{p_3(\log n)}.
    \]
    Then
    \[
       \limsup_{m,n-m \to \infty} \E[ \mathcal{C}_{3,m,n}^2 ] < \infty.
    \]
    \textcolor{red}{I expect this to follow from the concentration bounds for speed in ``Random walks in hyperbolic spaces...'' by Aoun and Sert. Probably not quite as immediate as bound on tail of $\mathcal{C}_1$,
    but I think should go through.}
\end{lem}

\begin{proof}[Proof of Theorem \ref{thm:lamplighterCLT} given Lemmas]
   \textcolor{red}{Again, this
   is assuming that the conclusion
   of the deterministic lemma
   has the form I wrote;
   write out that lemma
   carefully and adjust the 
   proof here/lemmas above as 
   needed.}
   Define a defective adapted
   cocycle by
   \[
      Q_n := |TSP(1,R_n,Z_n)|
   \]
   where $(R_n,Z_n) = \tilde{Z}_n$.
   By construction of the $\mathcal{C}_i$ and by
   Lemma \ref{lem:deterministic},
   we have that the defect satisfies
   \[
      |\Psi_{m,n}| \le
      (\mathcal{C}_{1,m,n} +
      \mathcal{C}_{2,m,n} +
      \mathcal{C}_{3,m,n})p_4(\log n).
   \]
   By Lemmas \ref{lem:boundC1},
   \ref{lem:boundC2}, and \ref{lem:boundC3}, there 
   is some constant $C$ (independent
   of $m$ and $n$) and some $N_0 \in \N$ such that
   whenever $m, m-n \ge N_0$
   we have that
   \[
      \E[ \mathcal{C}_{i,m,n}^2] \le C
   \]
   for $i=1,2,3$. Cauchy-Schwarz
   then tells us that for some $C'$,
   whenever $m, n-m \ge N_0$ we 
   have
   \[
      \E[|\Psi_{m,n}|^2] \le C' p_4(\log n)^2.
   \]
   Then applying Theorem \ref{thm: polylogarithmic MS CLT} gives our result.
\end{proof}
%\section{The CLT for the range of random walks on hyperbolic groups}

We borrow the framework from \cite{MathieuSisto2020} for proving a CLT - We observe the following trivial fact that whenever $ 1 \leq m \leq n $ and denoting $ R _{m,n} $ for the range between times $ m $ and $ n $ we have 
\[ |R _{n}| = |R _{m}| + |R _{m,n}| - |R _{m} \cap R _{m,n}|  .\]

In the language of \cite{MathieuSisto2020}, we say that $ \{ |R _{n}| \} _{n\geq 1} $ is a \emph{defective adapted cocycle} with defect $ \Phi _{m,n} := |R _{m} \cap R _{m,n}| $.

By theorem 4.2 in \cite{MathieuSisto2020}, to prove a CLT for the sequence $ |R _{n}| $ it is enough to show a second-moment deviation inequality: that 

\[ \mathbb{E} [\Phi _{m,n} ^{2} ] \leq C .\] 

For some $ C $ not depending on $ m,n $. We instead prove a stronger version of the deviation inequality:

\begin{prop}  There exists $ C>0 $ such that for any $ 1 \leq m \leq n $.
	\[ \mathbb{P}(\Phi _{m,n} \geq k) \leq Ce ^{-k/C} ,\]
\end{prop}
\begin{proof}
	Let $ \hat{R} _{n} $ denote the range of the reversed random walk - that is, the random walk driving by $ \hat{\mu} $.` If is enough to show that for any $ n, n' \in \mathbb{N} $ we have 
	\[ \mathbb{P}(\sup _{n, n'} |\hat{R} _{n} \cap R _{n} | \geq k) \leq Ce ^{-k/C} .\]
	
	This is an immediate consequence of lemma 5.3 of \cite{Choi2023deviation}. \textcolor{blue}{(maybe this is actually Lemma 4.9 of the arxiv version of \cite{Choi2023deviation}?)}
\end{proof}

This concludes the proof of Theorem \ref{thm: CLT range on hyperbolic groups}.

\bibliographystyle{alpha}
\bibliography{biblio.bib}
\end{document}
