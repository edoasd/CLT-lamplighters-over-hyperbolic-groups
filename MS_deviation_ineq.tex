\section{Mathieu-Sisto's deviation inequalities and consequences}
\subsection{Defective adapted cocycles and the central limit theorem}

A sequence $\mathcal{Q}=\{Q_n\}_{n\ge 1}$ of maps $Q_n:\Omega\to \R$ such that $Q_n$ is measurable with respect to $\sigma(X_1,\ldots, X_n)$, for each $n\ge 1$, is called a \emph{defective adapted cocycle}. We will use the convention $Q_0\equiv 0.$ The \emph{defect of $\mathcal{Q}$} is the collection of maps $\Psi=\{\Psi_{n,m}\}_{n,m\ge 0}$ defined by
\[
\Psi_{n,m}(w)=Q_{n+m}(w)-Q_n(w)-(Q_m\circ \theta^n)(w), \text{ for each }w\in \Omega \text{ and }n,m\ge 0.
\] 
The following result states that the central limit theorem holds for defective adapted cocycles that satisfy a second-moment deviation inequality.


\begin{thm}[{\cite[Theorem 4.2]{MathieuSisto2020}}] \label{thm: MS general CLT with constant deviation ineq}
	Let $G$ be a countable group endowed with a probability measure $\mu$. Consider $\mathcal{Q}$ a defective adapted cocycle on $\Omega=G^{\Z_{+}}$, and denote by $\{\Psi_{n,m}\}_{n,m\ge 0}$ its defect. Suppose that
	\begin{enumerate}
		\item $\E[|Q_1|^{2}]<\infty$, and
		\item 	$\sup_{m,n\ge 0} \left\{ \E\left[|\Psi_{n,m}|^2\right]\right\} <\infty$.
	\end{enumerate}
	
	Then, there exist constants $\ell,\sigma\in \R$ such that the random variables $\frac{1}{\sqrt{n}}\left(Q_n-\ell n\right)$ converge in law to a Gaussian random variable with zero mean and variance $\sigma^2$.
\end{thm}

Furthermore, it is proved in \cite[Theorem 3.3]{MathieuSisto2020} that the constant $\ell$ that appears in the statement of Theorem \ref{thm: MS general CLT with constant deviation ineq} satisfies that $\frac{1}{n}Q_n$ converges to $\ell$ in $L_1$ as $n\to \infty$.  


We will use this result for the defective adapted cocycle obtained from the word length of the $\mu$-random walk on $G$ at time $n$. That is, we will consider some word metric $d$ on $G$, and define $Q_n\coloneqq d(e_G,Z_n)$, for each $n\ge 1$. Since we will be working with finitely supported probability measures, it holds immediately that $\E[|Q_1|^2]<\infty$. 


The objective of the following sections of this paper is to prove that there exists a constant $C>0$ such that
\[
\E[d(e_G,Z_{n+m})-d(e_G,Z_n)-d(e_G,Z_m\circ \theta^n)]=\E[|\Psi_{n,m}|^2]\le C, \text{ for each }n,m\ge 0,
\]

where $G=A\wr H$, $d$ will be the switch-walk-switch word metric and $\mu$ is a finitely supported probability measure, as in the hypotheses of Theorem \ref{thm: main theorem CLT for word length for lamplighter over a hyperbolic base group}.

\subsection{Continuity results}
The following results are obtained in Sections 5 and 6 of Mathieu-Sisto paper.
\begin{enumerate}
	\item Continuity and differentiability of drift .
	\item Continuity and differentiability of asymptotic entropy.
\end{enumerate}
We will apply them to lamplighters over hyperbolic groups.
\subsection{Polylog deviation inequalities}

In this section we explain a generalization of \cite[Theorems 4.1 \& 4.2]{MathieuSisto2020} (see Theorem \ref{thm: MS general CLT with constant deviation ineq})
\red{remember to do the modification of n to m+n.}
\begin{thm} \label{thm: polylogarithmic MS CLT}
	Suppose that $Q_n$ is a defective adapted cocycle with defect
\[
\Psi_{n,m}(w)=Q_{n+m}(w)-Q_n(w)-(Q_m\circ \theta^n)(w), \text{ for each }w\in \Omega \text{ and }n,m\ge 0.
\] 
	Suppose that for some fixed polynomial $p$ and $N_0 \in \mathbb{N}$ we have that
	\[
	\E[|\Psi_{n,m}|^2] \le p(\log (n+m))
	\]
	whenever $n,m \ge N_0$.
	Then a CLT holds for $Q_n$.
\end{thm}

Here are the places in the proof of Mathieu-Sisto's CLT where the deviation inequality is used:
\begin{enumerate}
%	\item In Theorem 4.4 one obtains $V(Q_n)\le n\Big(p(4 \E(Q_1^2)+ 16\log(n))\Big)$
%	\item Then in Lemma 4.5 it is cited a result of Hammersley \cite[Theorem 2]{Hammerslet1962}. This should be replaced by an inequality of the form $a_{n+m}\le a_n+a_m+bp(log(n+m))$.
	\item The deviation inequality is used in multiple occasions during the proof of Lemma 4.6. One should do the appropriate modifications.
\end{enumerate}
The proof follows very similarly to that of \cite[Theorems 4.1\& 4.2]{MathieuSisto2020}. We now explain the parts of the proof where minor modifications are needed.

	The first step in the proof is \cite[Theorem 4.4]{MathieuSisto2020} and its application to obtain \cite[Theorem 4.2]{MathieuSisto2020}. With an analogous proof, one obtains 
	\[
	V(Q_n)\le n\left(p\E(Q_1^2)+16\log(n)\right),
	\]
	where $p$ is the polynomial from the second moment deviation inequality.
	
	The second part of the proof is in \cite[Lemma 4.5]{MathieuSisto2020}. Here, we can use \cite[Theorem 2]{Hammerslet1962} to obtain the following statement.
	\begin{lem}\label{lem: hammersley}
		Let $(a_n)_{n\ge 0}$ be a sequence of real numbers and let $p$ be a polynomial. Suppose that there is $b\ge 0$ such that
		\[
		a_{n+m}\le a_n+a_m+bp(\log(n+m)),\text{ for each }n,m\ge 1.
		\]
		Then $\lim_{n\to \infty}\frac{a_n}{n}$ exists.
	\end{lem}
	 The most important part of the proof is the following lemma. 
	 \begin{lem}
	 	Let $(Q_n)_n$ be a DAC with a finite second moment and that satisfies a second-moment deviation inequality. Then
	 	\[
	 	\lim_{M\to \infty}\limsup_{n\to \infty}\frac{1}{n}V\Big(Q_n-\sum_{j=0}^{\lceil n/2^M \rceil-1}Q_{2^M}\circ \theta_{j2^M}\Big)=0.
	 	\]
	 \end{lem}
	 \begin{proof}
	 	This should be done more carefully.
	 \end{proof}





