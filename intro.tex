\section{Introduction}



\begin{enumerate}
\item	Talk about random walks on groups; trying to prove limit laws in general
\item Make a list of things known about lamplighters and why they are relevant
\item Say that in this paper we concentrate on the CLT
\end{enumerate}
\blue{Here are some things that we should cite... I am missing many more but this is a good start.}
\cite{Salaun2001}.

\begin{enumerate}
	\item The CLT for non-abelian free groups is due to \cite{SawyerSteger1987} and \cite{Ledrappier2001}. Then for non-elementary hyperbolic groups with a finite exponential moment is due to \cite{Bjorklund2010}. This was generalized for any finite second moment measure in \cite{BenoistQuint2016hyperbolic}. The last two results hold more generally for group acting on a Gromov hyperbolic space by isometries. \cite{BenoistQuint2016} show a CLT for random walks on $\mathrm{GL}_d(\R)$ with a finite second moment. See also \cite{Gouezel2017}.
	\item \cite{ErschlerZheng2022} prove a law of large numbers for random walks on $\Z/2\Z\wr \Z^2$ with a finite $(2+\varepsilon)$-moment, for some $\varepsilon>0$. They also discuss limit laws in other wreath products.
	\item \cite{BahmanianForghaniGekhtmanMallahiKarai2024} prove a central limit theorem for random walks on the group of affine transformations of a horospherical product of Gromov hyperbolic spaces.
	\item \cite{Choi2023} proves a central limit theorem for groups acting with contracting elements.
	\item \cite{GekhtmanTaylorTiozzo2022} prove a CLT with respect to the counting measure on the Cayley graph of a group acting on a hyperbolic space.
	\item \cite{Horbez2018} proves a CLT for random walks on mapping class groups and $\mathrm{Out}(F_n).$
	\item \cite{LeBars2022} proves a CLT for groups acting on a $\mathrm{CAT}(0)$ space.
	\item  \cite{Gilch2008} proves that the drift of $\Z/2\Z\wr G$ is strictly larger than that of its projection to $G$.
	\item \cite{MrazovicSandriSebek2023} prove a LLN and CLT for the capacity of the range of a random walk on a group.
	\item \cite{Salaun2001} proves a LLN and CLT for a simple random walk on a free group, conditioned on the boundary point.
\end{enumerate}


\subsection{Main results}

Consider the switch-walk-switch word length $|\cdot |$ on $A\wr H$.

\begin{thm}
	Let $A$ be a non-trivial group and $H$ a non-elementary hyperbolic group. Consider a probability measure $\mu$ on $A\wr H$ such that $\mu_H$ is non-elementary and has a finite second moment, and such that $\mu$ has bounded lamp range. Denote by $\{w_n\}_{n\ge 0}$ the $\mu$-random walk on $A\wr H$, and let $C=\lim_{n\to \infty}\frac{\E(|w_n|)}{n}$ be the drift of the $\mu$-random walk on $A\wr B$. Then the sequence of normalized random variables $\frac{|w_n|-Cn}{\sqrt{n}}$, $n\ge 1$, converges in law to a non-degenerate gaussian law.
\end{thm}

\red{What are the moment hypotheses for acylindrically hyperbolic base groups? Finite support? exponential tails?}

We prove a central limit theorem for the range of random walks with a finite second moment on hyperbolic groups.
\begin{thm}\label{thm: CLT range on hyperbolic groups}
	Let $H$ be a non-elementary hyperbolic group and let $\mu$ be a non-elementary probability measure on $H$ with a finite second moment. Let $C$ be the probability that the $\mu$-random walk on $H$ starting at $e_H$ never returns to $e_H$. Then the sequence of normalized random variables $\frac{|R _{n} | - Cn}{\sqrt{n}}$, $n\ge 1$, converges in law to a non-degenerate gaussian law.
\end{thm}
