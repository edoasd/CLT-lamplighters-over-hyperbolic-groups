\usepackage[T1]{fontenc}
\usepackage[utf8]{inputenc}
\usepackage{lmodern}
\usepackage[french,english]{babel}
\usepackage{geometry} 
%\geometry{a4paper}

\geometry{%
	a4paper,                % format de papier
	% Définition des marges :
	left= 3cm,            % marge intérieure à la page
	right = 3cm,          % marge extérieure
	top = 3cm,
	bottom = 3cm,
	% En-tête et pied de page :
	headheight=6mm,         % espace réservé à l'en-tête dans la marge top
	headsep=9mm,            % espace entre le corps et l'en-tête
	footskip=9mm            % espace entre le corps et le pied de page
}
\usepackage{amsmath,amssymb,amsfonts,amsthm,mathtools}
\usepackage{fancyhdr}
\usepackage{emptypage}
\pagestyle{fancy}
\renewcommand{\sectionmark}[1]{\markright{\thesection.\ #1}}
\fancyhf{}
\fancyhead[LE]{\leftmark}
\fancyhead[RO]{\rightmark}
\fancyfoot[C]{\thepage}

\fancypagestyle{plain}{
	\fancyhf{}
	\fancyfoot[RO,RE]{\thepage}
	\renewcommand{\headrulewidth}{0pt}
	\renewcommand{\footrulewidth}{0pt}}
\newcommand*{\img}[1]{%
	\raisebox{-.0\baselineskip}{%
		\includegraphics[
		height=\baselineskip,
		width=\baselineskip,
		keepaspectratio,
		]{#1}%
	}%
}
\newcommand\blfootnote[1]{%
	\begingroup
	\renewcommand\thefootnote{}\footnote{#1}%
	\addtocounter{footnote}{-1}%
	\endgroup
}
\usepackage{comment} 
\usepackage{xcolor}

\usepackage[noadjust]{cite}
\usepackage[
pagebackref=true,
colorlinks=true,
urlcolor=purple,
linkcolor=purple!87!black,
citecolor=green!60!black,
pdfborder={0 0 0}
]{hyperref}
\renewcommand*{\backref}[1]{}
\renewcommand*{\backrefalt}[4]{[{\tiny%
		\ifcase #1 Not cited.%
		\or Cited on page~#2.%
		\else Cited on pages #2.%
		\fi%
	}]}
\usepackage{bookmark}
\usepackage{dsfont}
%% Colors
\newcommand{\red}[1]{\textcolor{red}{#1}}
\newcommand{\blue}[1]{\textcolor{blue!60!black}{#1}}
%% Common notation
\newcommand{\id}{\mathsf{id}}
\newcommand{\supp}[1]{\mathrm{supp}(#1)}
\newcommand{\cay}[2]{\mathrm{Cay}(#1,#2)}
\newcommand{\hceil}[1]{\Big\lceil #1 \Big\rceil}
\newcommand{\hfloor}[1]{\Big\lfloor #1 \Big\rfloor}
%% Numbers: Z, R, C, Q
\newcommand{\Z}{\mathbb{Z}}
\newcommand{\R}{\mathbb{R}}
\newcommand{\C}{\mathbb{C}}
\newcommand{\Q}{\mathbb{Q}}
\newcommand{\N}{\mathbb{N}}
%% Probabilty: Proba, expectation
\renewcommand{\P}{\mathbb{P}}
\newcommand{\E}{\mathbb{E}}
\newcommand{\Prob}{\mathbb{P}}
\newcommand{\tsp}{\mathrm{TSP}}


\newenvironment{questionbis}[1]
{\addtocounter{thm}{-1}
	\renewcommand{\thethm}{\ref*{#1}$'$}%
	\neutralize{question}\phantomsection
	\begin{question}}
	{\end{question}}

\newcommand{\thistheoremname}{}

\newtheorem*{genericthm*}{\thistheoremname}
\newenvironment{namedthm*}[1]
{\renewcommand{\thistheoremname}{#1}%
	\begin{genericthm*}}
	{\end{genericthm*}}
\theoremstyle{plain}
\newtheorem{thm}{Theorem}[section] % reset theorem numbering for each chapter
\newtheorem{prop}[thm]{Proposition}
\newtheorem{lem}[thm]{Lemma}
\newtheorem{cor}[thm]{Corollary}
\theoremstyle{definition}
\newtheorem{rem}[thm]{Remark}
\newtheorem{defin}{Definition}
\newtheorem{exmp}[thm]{Example} % same for example numbers
\newtheorem{question}[thm]{Question}
\newtheorem{conjecture}[thm]{Conjecture}
\newtheorem{exercise}[thm]{Exercise}
\newtheorem{condition}{Condition}
